
%% bare_conf.tex
%% V1.2
%% 2002/11/18
%% by Michael Shell
%% mshell@ece.gatech.edu
%%
%% NOTE: This text file uses MS Windows line feed conventions. When (human)
%% reading this file on other platforms, you may have to use a text
%% editor that can handle lines terminated by the MS Windows line feed
%% characters (0x0D 0x0A).
%%
%% This is a skeleton file demonstrating the use of IEEEtran.cls
%% (requires IEEEtran.cls version 1.6b or later) with an IEEE conference paper.
%%
%% Support sites:
%% http://www.ieee.org
%% and/or
%% http://www.ctan.org/tex-archive/macros/latex/contrib/supported/IEEEtran/
%%
%% This code is offered as-is - no warranty - user assumes all risk.
%% Free to use, distribute and modify.

% *** Authors should verify (and, if needed, correct) their LaTeX system  ***
% *** with the testflow diagnostic prior to trusting their LaTeX platform ***
% *** with production work. IEEE's font choices can trigger bugs that do  ***
% *** not appear when using other class files.                            ***
% Testflow can be obtained at:
% http://www.ctan.org/tex-archive/macros/latex/contrib/supported/IEEEtran/testflow


% Note that the a4paper option is mainly intended so that authors in
% countries using A4 can easily print to A4 and see how their papers will
% look in print. Authors are encouraged to use U.S. letter paper when
% submitting to IEEE. Use the testflow package mentioned above to verify
% correct handling of both paper sizes by the author's LaTeX system.
%
% Also note that the "draftcls" or "draftclsnofoot", not "draft", option
% should be used if it is desired that the figures are to be displayed in
% draft mode.
%
% This paper can be formatted using the peerreviewca
% (instead of conference) mode.
\documentclass[conference]{IEEEtran}


% correct bad hyphenation here
%\hyphenation{op-tical net-works semi-conduc-tor IEEEtran}


%-------------------------------------------------------------------------
% take the % away on next line to produce the final camera-ready version
%\pagestyle{empty}
\pagenumbering{arabic}
\usepackage{verbatim}

\begin{comment}
%--------------- added by honggang, to make it tighter---------------
\topskip 0in            %between header and text
\parskip 0in            %gap between paragraphs
\floatsep 0in           %space left between floats
\textfloatsep 0.01in       %space between last top float or first bottom float and the text
\intextsep 0.01in          %space left on top and bottom of an in-text float
\dblfloatsep 0in        %floatsep for 2 column
\dbltextfloatsep 0in    %space between last top float or first bottom float and the text
%\abovecaptionskip 0in   %space above caption
%\belowcaptionskip 0in   %space below caption
\abovedisplayskip 0in   %space before maths
\belowdisplayskip 0in   %space after maths
\arraycolsep 0in       %gap between columns of an array
\topsep 0in             %space between first list item and preceding paragraph.
\partopsep 0in          %extra space added to \topsep when environment
                        %starts a new paragraph.
\itemsep 0in            %space between successive list items
%----------------------- added by honggang, to make it tighter--------
\end{comment}


\usepackage{times}
\usepackage{latexsym}
\usepackage{amstext}
\usepackage{amssymb}
\usepackage{amsmath}




\def\@begintheorem#1#2{\tmpitemindent\itemindent\topsep 0pt\rmfamily\trivlist
    \item[\hskip \labelsep{\indent\itshape #1\ #2:}]\itemindent\tmpitemindent}
\def\@opargbegintheorem#1#2#3{\tmpitemindent\itemindent\topsep 0pt\rmfamily \trivlist
% IEEE uses a long dash between the theorem number and name and does not
% put () around the name
%    \item[\hskip\labelsep{\indent\itshape #1\ #2\ \rmfamily(#3)}]\itemindent\tmpitemindent}
    \item[\hskip\labelsep{\indent\itshape #1\ #2\rmfamily---#3:}]\itemindent\tmpitemindent}
\def\@endtheorem{\endtrivlist\unskip}


\newtheorem{theorem}{Theorem}
\newtheorem{lemma}{Lemma}
\newtheorem{corollary}{Corollary}
\newtheorem{definition}{Definition}
\newtheorem{claim}{Claim}
\newtheorem{proposition}{Proposition}

\def\QED{\mbox{\rule[0pt]{1.5ex}{1.5ex}}}
\def\proof{\noindent\hspace{2em}{\itshape Proof: }}
\def\endproof{\hspace*{\fill}~\QED\par\endtrivlist\unskip}


%
%  dubfigsingle.  2x1 double figure with only one \caption.
%

\newcommand{\dubfigsingle}[6]{
%\vspace*{-0.2in}
\begin{figure*}[t]
\centerline{
    \begin{minipage}{0.55\textwidth}
      \begin{center}
        \leavevmode
        \setlength{\epsfxsize}{0.9\textwidth}
        \setlength{\epsfysize}{0.59\textwidth}
        \epsffile{\Figdir#1}
       \newline{\small (a) #2}
      \end{center}
    \end{minipage}
    \begin{minipage}{0.55\textwidth}
      \begin{center}
        \leavevmode
        \setlength{\epsfxsize}{0.9\textwidth}
        \setlength{\epsfysize}{0.59\textwidth}
        \epsffile{\Figdir#3}
       \newline{\small (b) #4}
      \end{center}
    \end{minipage}
} \caption{#5}\label{fig:#6}
%\hrule
%\vspace*{-0.2in}
\end{figure*}
}


%\input{epsf}

\usepackage{epsfig}
\usepackage{times}
\usepackage{amsfonts}
\usepackage{amsmath}
\usepackage{latexsym}
\usepackage{amssymb}



\begin{document}

\title{On Unstructured File Sharing Networks
%\\
%Multipath and Multiple TCP Connections
}


\author {
 \normalsize
\begin{tabular}{ccc}
\multicolumn{3}{c}{ \large{Honggang Zhang$^*$, Giovanni
Neglia$^{\dag}$, Don Towsley$^*$, Giuseppe Lo Presti$^\S$}
} \\
 $^*$Computer Science Dept.  & $^{\dag}$D.I.E.E.T.
& $^\S$IT Dept.\\
 University of Massachusetts Amherst& Universit\`a degli Studi di Palermo, Italy &
 C.E.R.N., Switzerland\\
\{honggang, towsley\}@cs.umass.edu & giovanni.neglia@ieee.org
& giuseppe.lopresti@cern.ch\\
\end{tabular}\\
\\
\large{\textbf{UMass Amherst Computer Science Department Technical Report 2006-40}}\\
First version: August 1, 2006\\
This version: August 15, 2006}
%\large{Honggang Zhang} & \large{Giovanni Neglia} &
%\large{Don Towsley}


%\author{Honggang Zhang\\
%honggang@cs.umass.edu\\
%\today
%}



\maketitle
%\thispagestyle{empty}



\begin{abstract}

We study the interaction among users of unstructured file sharing
applications, who compete for available network resources (link
bandwidth or capacity) by opening multiple connections on multiple
paths so as to accelerate data transfer. We model this interaction
with an \textit{unstructured file sharing game}. Users are players
and their strategies are the numbers of sessions on available
paths. We consider a general bandwidth sharing framework proposed
by Kelly \cite{kelly98rate} and Mo and Walrand \cite{mo00fair},
with TCP as a special case. Furthermore, we incorporate the
Tit-for-Tat strategy (adopted by BitTorrent \cite{bt} networks)
into the unstructured file sharing game to model the competition
in which a connection can be set up only when both users find this
connection beneficial.
%This strategy is a widely known strategy built
%into BitTorrent \cite{bt} networks.
%Whether a connection setup is successful or not depends on whether
%both peers find this connection beneficial or not. To make our
%model more tractable, we restrict users to open either zero or one
%connection to another peer.
We refer to this as an \textit{overlay formation game}. We prove
the existence of Nash equilibrium in several variants of both
games, and quantify the losses of efficiency of Nash equilibria.
We find that the loss of efficiency due to selfish behavior is
still unbounded even when the Tit-for-Tat strategy is believed to
prevent selfish behavior.
%can be arbitrarily large if users are not resource constrained.
%Fortunately, the loss of efficiency is upper bounded when users
%are resource constrained.
%Potential
%solutions are recommended to improve the system efficiency (to be
%done !!!).
\end{abstract}



% no keywords

% For peer review papers, you can put extra information on the cover
% page as needed:
% \begin{center} \bfseries EDICS Category: 3-BBND \end{center}
%
% for peerreview papers, inserts a page break and creates the second title.
% Will be ignored for other modes.
\IEEEpeerreviewmaketitle



%\input{tcpgame-general-intro}
%\parskip=-5pt
\section{Introduction}
%\parskip=0pt


\noindent Recently peer-to-peer applications (e.g., BitTorrent
\cite{bt}, Kazaa, eDonkey, and Gnutella \cite{p2p}) have become
very popular. They can be major contributors of the Internet
traffic. For example, Sprint's IP Monitoring Project \cite{sprint}
shows that in April $2003$, $20-40\%$ of total bytes corresponded
to peer-to-peer traffic on one backbone link. CacheLogic
\cite{cachelogic} estimates that peer-to-peer generated $60\%$ of
all US Internet traffic at the end of $2004$.
%ISPs
%cannot afford to ignore peer-to-peer traffic as more than $92\%$
%of p2p traffic cross transit or peer links. On average, $80\%$ of
%upstream capacity is consumed by p2p.

We refer to the networks for these peer-to-peer applications as
unstructured file sharing overlay networks. These networks are
overlay networks since users forward or relay traffic for each
other. These networks are also \textit{unstructured} because there
are no well-defined network topologies, and users are not under
the control of some central entity. For comparison,
%Akamai
%\cite{akamai} and
Resilient Overlay Network \cite{andersen01resilient} is a
\textit{structured} overlay network.
%As more content providers use peer-to-peer applications
%to deliver content, their costs will be passed onto service
%providers.
Given the increasingly large share of Internet traffic from
unstructured file sharing networks, it is important to understand
the behavior and performance of such networks, and such a
fundamental understanding will certainly help ISPs and aid in the
design of future Internet architecture.

In this paper, we investigate the strategic behavior of
self-interested peers/users of such unstructured file sharing
overlay networks. Our work differs from  previous works on
peer-to-peer applications, whose focus are on file searching and
replication \cite{cohen02replication}, and topology discovery
\cite{stutzbach05p2p}. Specifically, our investigations are from
two different angles.

First, we study the interaction among users of unstructured file
sharing applications, who compete for available network resources
(link bandwidth or capacity) by opening multiple connections or
sessions on multiple paths so as to accelerate data transfer. We
introduce an \textit{unstructured file sharing game} to model this
interaction. In this game, users are players and their strategies
are the numbers of sessions on available paths. The data rate
allocated to connections are determined by the network. The
mechanism of rate allocation considered by us is a general
bandwidth sharing framework proposed by Kelly \cite{kelly98rate}
with TCP networks as special cases
\cite{bonald01impact}\cite{mo00fair}. Our focus is on TCP networks
in which all connections/sessions are TCP connections. The
unstructured file sharing game generalizes the \textit{TCP
connection game} introduced in \cite{zhang05tcpgame_icnp} where
the competition for a single bottleneck link capacity is
investigated.

Second, we incorporate the Tit-for-Tat strategy into the
unstructured file sharing game. This strategy is widely known and
built into BitTorrent \cite{bt} networks. With this strategy,
peers set up a connection between themselves only when they both
find it beneficial. We model this interaction scenario as an
\textit{overlay formation game}. In order to make our model
tractable, we restrict users to open either zero or one connection
to another peer.
%A user makes a decision on which peer among
%multiple other peers to open a connection, with the objective to
%maximize its benefit. However,

In both games, users are interested in maximizing their benefits,
a combination of some utility function and the cost associated
with maintaining data transfer sessions. We assume that utility
functions are increasing and concave functions of the data
throughput in bits per second. Throughput is defined as the
successful packet delivery rate. The cost incurred to users
includes memory cost and CPU cost. As in
\cite{zhang05tcpgame_icnp}, we consider a cost that is
proportional to the total number of connections opened by a user.
We also consider another type of cost which is proportional to a
user's packet sending rate.

We are interested in the following questions. First, does there
exist a stable network state (i.e., Nash equilibrium (NE)
\cite{basar98game}) in both games?  If so, what is the system
performance at a NE? Specifically, we are interested in the loss
of efficiency of a NE and the price of anarchy
\cite{koutsoupias99worstcase} of NE(s). The loss of efficiency of
a NE is defined as the ratio of the optimal system performance
over the system performance at the NE, and the worst loss of
efficiency is referred to as the price of anarchy
\cite{koutsoupias99worstcase}. These metrics capture how bad the
competition can be among self-interested TCP users. Here we focus
on pure strategy NE.

We make the following contributions.

First, we give a formal formulation of unstructured file sharing
game, and show by examples that multiple NEs exist on general
network topologies. We then focus on parallel link networks and
star networks, which are used to model peer-to-peer applications
(similar topologies were also studied in
\cite{qiu04bt}\cite{piccolo04p2p}). We prove the existence of NE
of unstructured file sharing games on both networks, and find
that, if users are not resource constrained, the efficiency loss
of NEs can be unbounded (i.e., price of anarchy is arbitrarily
large). Fortunately, if there are resource constraints for users,
the efficiency loss is upper bounded. We also demonstrate the
stability of NE in best-response dynamics in several variants of
the game.

Second, we model the Tit-for-Tat strategy in unstructured file
sharing networks by an overlay formation game. We show
analytically the existence of equilibrium overlay networks and
that the loss of efficiency can be arbitrarily large. Tit-for-Tat
is believed to prevent selfish behavior. However, our results show
that the loss of efficiency due to selfish behavior can still be
unbounded.
%Our numerical simulations further verify our analytical
%results.

The rest of this paper is organized as follows. Related work is
presented in Section \ref{sec:related}. The problem formulation
for unstructured file sharing game is given in Section
\ref{sec:form}. In Sections \ref{sec:para} and \ref{sec:star}, we
focus on unstructured file sharing game on a parallel link network
and star network. We address the overlay formation game in Section
\ref{sec:topo}. Conclusions are given in Section \ref{sec:con}.

\begin{comment}

We first demonstrate that there exists a unique Nash equilibrium
in the unstructured file sharing game when user's benefit includes
only utility. At the Nash equilibrium, each user opens maximum
allowable number of connections on each available path.
Furthermore, the price of anarchy can be unbounded if users can
open arbitrarily large number of data transfer sessions.

If user's benefit includes both utility and cost, then we
demonstrate through simple examples that there might exist
multiple Nash equilibria in unstructured file sharing games.

Then, we focus on two specific network topologies: parallel link
topology and star topology. We show that Nash equilibrium exists
for the games on these two kinds of topologies, and there are
losses of efficiency at those Nash equilibria.

(.....)

(... network formation game using tit-for-tat ...)
\end{comment}



\begin{comment}
Then, we suppose that all users of such kind of overlay network
use multi-path congestion controller recently proposed by \cite{}
and \cite{}. We demonstrate in Section \ref{sec:un-coord} that
multi-path congestion controller is able to improve the average
throughtput of such kind of networks when compared with the
multiple un-coordinated congestion controllers which are adopted
by BitTorrent networks.
\end{comment}

%\input{tcpgame-general-related}

%\parskip=-15pt

\section{Related Work}\label{sec:related}

%\parskip=0pt

Johari \textit{et al} \cite{johari04allocation} study a congestion
game where users of a congested resource anticipate the effect of
their actions on the price of the resource.
%For a general network
%topology, they show that the selfish behavior of the users leads
%to an aggregate utility which is no worse than 3/4 the maximum
%possible aggregate utility.
In \cite{johari04allocation} users compete for each link
independently from other links in the network. But this
independence characteristic is not true for our model, because if
a user opens a connection on a path, then all links of this path
must carry this connection. \cite{akella02selfish} and
\cite{zhang05tcpgame_icnp} study the interactions among selfish
TCP users competing for a single bottleneck link. The unstructured
file sharing game in this paper can be thought of as a generalized
version of the game in \cite{zhang05tcpgame_icnp}.
%Morris and Tay \cite{morris03user}
%study the influence of user behavior on congestion collapse
%through a Markovian model.

%Several recent studies on congestion control
\cite{kelly05joint}\cite{han07overlay_tcp} propose multi-path
congestion controllers by which users can coordinate the data
transfer sessions on several different paths to improve data
throughput. A multi-path congestion controller chooses rates at
which to send data on all of the paths available to it.
%is a coordinated
%controller where all sessions on different paths coordinate with
%each other and all sessions act together like a single congestion
%controller.
%However,
In our models, all sessions controlled by a single user are
independent congestion controllers.
%The Tit-for-Tat strategy was studied in the past
%\cite{srinivasan03cooperation}\cite{qiu04bt}.
%This strategy is
%generally considered robust as analytically supported
%by~\cite{qiu04bt}.
\cite{qiu04bt} studies how Tit-for-Tat affects selfish peers who
are able to set their uploading bandwidth. Our work differs from
\cite{qiu04bt} in that we assume that a user can benefit by
changing the number of connections to open. The analytical
framework for our overlay formation game is in~\cite{jackson04}.

%????? many research on p2p focused on searching, replication
%techniques, topology discovery... Not the focus of this paper, ...



\begin{comment}
\begin{eqnarray}
\sum_{m=1}^i\sum_{n=1}^js_{mi}q_{mn}s_{nj} &=&
\big(\frac{-1}{\delta}\big)^{i+j}\sum_{m=1}^i\sum_{n=1}^j(-1)^{i+j}{i\choose
m}{j \choose n}q_{mn}\nonumber\\
& = &
\big(\frac{-1}{\delta}\big)^{i+j}\sum_{m=1}^i\sum_{n=1}^j(-1)^{i+j}{i\choose
m}{j \choose n} \Big\{E_\theta
\big[\frac{f(Y|\theta+m\delta)f(Y|\theta+n\delta)}{f^2(Y|\theta)}\big]-1\Big\}
\nonumber\\
&=&
\big(\frac{-1}{\delta}\big)^{i+j}\sum_{m=1}^i\sum_{n=1}^j(-1)^{i+j}{i\choose
m}{j \choose
n} \Big\{ E_{\theta}\Big[\frac{f(Y|\theta+m\delta)f(Y|\theta+n\delta)}{f^2(Y|\theta)}\Big]\nonumber\\
&&\rightarrow E_\theta[\frac{(\partial^{i}f(X,\theta)/
\partial\theta^i)(\partial^{j}f(X,\theta)/\partial\theta^j)}{f^2(X,\theta)}]=w_{ij}(\theta),
\nonumber
\end{eqnarray}

\end{comment}


%\input{tcpgame-general-formulation}
%\parskip=-10pt
\section{Unstructured File Sharing Game}\label{sec:form}

%In this section, we formally introduce the model of
%\textit{unstructured file sharing game}.
%\parskip=0pt
\subsection{Formulations}

%\parskip=0pt
Consider a network consisting of $J$ links, numbered $1, ..., J$.
Link $j$ has a capacity given by $C_j>0$; we let $\mathbf{C}=(C_1,
C_2, ..., C_J)$ denote the vector of capacities. A set of users
$\{1,..., R\}$ share this network. We assume that there exists a set
of paths through the network, numbered $1,...,P$. By an abuse of
notation, we will use $J,N,P$ to also denote the sets of links,
users, and paths, respectively. Each path $p\in P$ uses a subset of
the set of links $J$; if link $j$ is used by path $p$, we will denote
this by writing $j\in p$. Each user $r\in \mathbf{R}$ has a
collection of paths available through the network; if path $p$ serves
user $r$, we will denote this by writing $p\in r$.
%We will
%assume without loss of generality that paths are uniquely
%identified with users, so that for each path $p$ there exists a
%unique user $r$ such that $p\in r$. If two users happen to share
%the same path, that is captured in our model by creating two paths
%which use exactly the same subset of links.

\begin{comment}
Links required by individual paths are captured by the path-link
incidence matrix $A$, defined by:
\begin{equation}
A_{jp}= \left \{
\begin{array}{ll}
1, & \textrm{if $j\in p$}\\
0, & \textrm{if $j\notin p$}
\end{array} \right.
\end{equation}
Furthermore, we can capture the relationship between paths and
users by the path-user incidence matrix $H$, defined by:
\begin{equation}
H_{rp}= \left \{
\begin{array}{ll}
1, & \textrm{if $p\in r$}\\
0, & \textrm{if $p\notin r$}
\end{array} \right.
\end{equation}
%%%Note that by our assumption on paths, for each path $p$, we have
%%%$H_{rp}=1$ for exactly one user $r$.
%For convenience, we will use $P_r$ to represent the set of paths
%belong to user $r$. We will also use $P_r$ to denote the size of
%set $P_r$.
\end{comment}

Each user can open a number of concurrent connections $n_{rp}$ on
each path $p$ with $p\in r$. This defines a strategy vector for
user $r$ as $\mathbf{n}_r=(n_{rp})$ with $p\in P$ and $p\in r$.
Then a composite strategy vector of all users is given by
$\mathbf{n}=(\mathbf{n}_1, ..., \mathbf{n}_R)$.
%Based on our
%definition, each path is associated with only one user, so,
%$\mathbf{n}$ is a $P\times 1$ column vector.
For a given $\mathbf{n}$, a certain rate allocation mechanism
allocates a traffic rate $y_p$ to each connection on path $p$. We
will discuss rate allocation mechanisms in the following section.
For now, we simply state that, $\forall p\in P$, $y_p$ is a
function of $\mathbf{n}$. We use vector $\mathbf{y}=(y_p, p\in P)$
to represent a rate allocation on all paths.

The total date rate or throughput $G_r$ obtained by a user $r$ is:
%\begin{equation}
$G_r(\mathbf{n}_r)=\sum_{p\in r} n_{rp} y_p
\label{eqn:throughput}$,
%\end{equation}
where $n_{rp}$ is the number of connections opened by user $r$ on
path $p$. As  $y_p$ ($\forall p\in P$) is a function of
$\mathbf{n}$, the throughput of user $r$ is a function of the
number of connections of all users, namely, $G_r=f(\mathbf{n})$.
%In addition, for convenience, we can also use a matrix to
%represent rate allocation. Specifically, let $\mathbf{Y}$ be a
%$P\times P$ diagonal matrix with element $Y_{(p,p)}=y_p$ and
%$Y_{(p,k)}=0, \forall p\ne k$. $\mathbf{Y}$ is different way to
%represent rate vector $\mathbf{y}=(y_p, p\in P)$. We have
%$$\mathbf{H \cdot Y \cdot n =G}.$$
Any feasible rate allocation $\mathbf{y}$ must satisfy the
capacity constraint:
%\begin{equation}
$\sum_{r\in \mathbf{R}} \sum_{p: j\in p} n_{rp} y_p \le C_j, j\in J.$
%\label{eqn:constraint}
%\end{equation}
%Using matrix $\mathbf{A}$, we may write this constraint as
%$$\mathbf{A \cdot Y \cdot n \le C}.$$

We assume that user $r$ receives a utility $U_r(G_r)$ when
obtaining throughput $G_r$. We assume that $U_r$ is a continuous,
concave, and non-decreasing function of $G_r$, with domain $G_r
\ge 0$. A user $r$ has some cost $\Phi_r(\mathbf{n}_r)$ associated
with opened connections. We assume that this cost is proportional
to the total number of connections opened by this user on all its
available paths:
%\begin{equation}
$\Phi_r(\mathbf{n}_r) = \beta \sum_{p\in r} n_{rp}.$
%\label{eqn:cost1}
%\end{equation}
Note that $\beta \in [0,1]$, and it is interpreted as the
aggressiveness coefficient. Smaller $\beta$ corresponds to more
powerful computation resources. This type of cost is also
considered in \cite{zhang05tcpgame_icnp}. In general, we can
assume that $\Phi_r$ is a continuous, convex, and non-decreasing
function of $\mathbf{n}_r$.
%This general cost function is our
%future research topic.
The payoff or benefit of a user $r$ is a linear combination of
utility $U_r$ and cost $\Phi_r$, defined as:
\begin{equation}
B_r(\mathbf{n}_r)=U_r(\mathbf{n}_r) -
\Phi_r(\mathbf{n}_r).\label{eqn:benefit}
\end{equation}

\begin{comment}
%For an unstructured overlay network,
The system optimization problem is to maximize the average payoff
for all users of this overlay network. Formally, this problem is
defined as:
\begin{eqnarray}
SYSTEM \nonumber \\
&\mbox{maximize} & \sum_r B_r(d_r)    \\
&\mbox{subject to} & \mathbf{Ay\le C} \\
&& \mathbf{Hy=G} \\
&& y_p\ge 0, p\in P.
\end{eqnarray}
Since the objective function is continuous and the feasible region
is compact, an optimal solution $y$ exists. Since the feasible
region is also convex, if the functions $B^r$ are strictly
concave, then the optimal vector $\mathbf{G=Hy}$ is uniquely
defined (though $\mathbf{y}$ may not be unique).
\end{comment}

%\parskip=-5pt

\subsection{Rate Allocation Mechanism}\label{sec:rate.alloc}
%\parskip=0pt


We assume that the network allocates data rates to connections
based on the $\alpha$-bandwidth allocation scheme
\cite{bonald01impact}\cite{kelly98rate}\cite{mo00fair}:
\begin{eqnarray}
\mbox{maximize}_{\mathbf{y}}
& \sum_p w_p n_p^\alpha \frac{(y_p n_p)^{(1-\alpha)}}{1-\alpha} \label{eqn:rate.alloc}\\
\mbox{subject to} & \sum_{r\in \mathbf{R}} \sum_{p: j\in p} n_{rp}
y_p \le
C_j, j\in J  \\
& n_p=\sum_{r: p\in r} n_{rp} \forall p \in P.
\end{eqnarray}
where $w_p$ is the weight of path $p$. $n_p$ is the number of
connections or sessions on path $p$. Different values of $\alpha$
give different rate allocations. For example, as $\alpha \to
\infty$, this allocation mechanism corresponds to Max-Min
fairness. Rate allocation in a TCP network is well approximated
with $\alpha=2$ and $w_p=1/(RTT_p)^2$. Here, $RTT_p$ is the Round
Trip Time (RTT) of path $p$.

In a single link case and where all paths have the same RTT, this
$\alpha$-bandwidth allocation is simplified to a \textit{simple
rate allocation mechanism}. That is, for a link shared by $n$
flows with the same RTT, each flow or connection gets an equal
share of the bandwidth of the link, namely,
\begin{equation}
y= C /n. \label{eqn:prop}
\end{equation}
Thus if a user $r$ has $n_{r}$ flows, then its throughput $G_{r}$
is:
\begin{equation}
G_{r}(\mathbf{n}_r)= \left \{
\begin{array}{ll}
C n_{r}/\sum_{w\in \mathbf{R}} n_{w}, &\quad \textrm{if $n_{r}>0$}\\
0 , &\quad \textrm{otherwise}
\end{array} \right.
\end{equation}

\bigskip
\noindent \textbf{Remarks.} Note that this \textit{simple rate
allocation mechanism} cannot be extended to a network setting.
%Please see \cite{zhang06tcpgame_general} for more details.
Specifically, after we calculate the rate allocated to each user on
each link according to (\ref{eqn:prop}), we cannot simply say that
the allocated rate on a path can be given by $y_{rp}=
\mbox{min}_{j\in p} y_{rj}, \forall r\in \mathbf{R}.$ An illustrative
example is given in Appendix \ref{appd:simple}.


Note that the authors of \cite{johari04allocation} can use this
rate allocation mechanism because in their case, users compete for
each link independently from other links. However, in our case,
links can not be treated independently, as all links of a path
must carry the connections opened on this path. As shown in the
following section, this requirement makes the throughput of a user
neither a concave nor convex in the numbers of connections opened
by this user.
%This is illustrated in an example of grid network topology.
Thus, it is difficult to apply the existing game-theoretic results
(which requires concavity of utility functions) to the
unstructured file sharing game on general network topology. Thus,
in this paper we focus on two specific networks: parallel links
and a star.

%\parskip=-5pt

\subsection{Unstructured File Sharing Game}
%\parskip=0pt

Based on the previous formulations, we now introduce an
\textit{unstructured file sharing game}.
% to model the interaction
%among multiple users who use multiple concurrent connections on
%multiple paths to compete for the total network bandwidth.
In this game, each user $r$ tries to maximize its aggregate
benefit $B_r$ by adjusting $\mathbf{n}_r$, its number of
connections on its available paths. Namely, a user $r$ tries to
solve the following optimization problem:
\begin{eqnarray}
\mbox{max}_{\mathbf{n}_r} &  B_r(\mathbf{n}_r,
\mathbf{y^*}(\mathbf{n}_r))
\label{eqn:usr.opt}\\
\mbox{s.t.} & n_{r_p} \in [0, n_{r_p}^{max}], & \forall r_p \in P_r \label{eqn:usr.ss}\\
 &  \mathbf{y^*} = \mbox{argmax}_\mathbf{y} &
\sum_p w_p n_p^\alpha \frac{(y_p
 n_p)^{(1-\alpha)}}{1-\alpha} \label{eqn:usr.opt.alloc}\\
& \mbox{s.t.} & \sum_{r\in \mathbf{R}} \sum_{p: j\in p} n_{r_p} y_p
\le C_j, j\in J \nonumber \\
&  & n_p=\sum_{r: p\in r} n_{r_p}, \forall p \in P \nonumber
\end{eqnarray}
The decision variables of user $r$ is given by vector
$\mathbf{n}_r$. The set of available paths of user $r$ is
represented by $P_r$. (\ref{eqn:usr.opt.alloc}) indicates that the
throughput of each connection on a path is the solution of the
optimization problem defined in (\ref{eqn:rate.alloc}).
%(\ref{eqn:usr.opt.con}) represents the capacity constraints.
If $\alpha=2$ and $w_p=1/(RTT_p)^2$ and the network is a single
bottleneck link, this game becomes the TCP connection game
\cite{zhang05tcpgame_icnp}.

For a general network, we cannot obtain an explicit form of
function $B_r(\mathbf{n}_r)$ because there is no closed form
solution for the rate allocation problem
(\ref{eqn:usr.opt.alloc}). However, as shown later, we can obtain
an explicit form of $B_r(\mathbf{n}_r)$ for some specific networks
such as grid network, parallel link, and star network.

In fact, (\ref{eqn:usr.opt}) is a Bi-level Programming problem
which in general is NP-hard \cite{vicente94bilevel}. In this
paper, we do not try to obtain a general solution for
(\ref{eqn:usr.opt}) for each user. Instead, we focus on some
special network topologies for which there exist analytically
tractable and closed form solutions to (\ref{eqn:usr.opt.alloc}),
and for these networks, we investigate the existence of Nash
equilibrium.

Let $\mathbf{n}^*_r$ represent the solution to user $r$'s
optimization problem defined above. Formally, we have
$\mathbf{n}^*_r = \mbox{argmax}_{\mathbf{n}_r}  B_r(\mathbf{n})$.
A Nash equilibrium (NE) is defined as a composite strategy profile
or a vector of connections of all users, and no user can gain by
unilaterally deviating from it. We denote a Nash equilibrium by:
%\begin{equation}
$\mathbf{n}^*=(\mathbf{n_1}^*, \mathbf{n_2}^*, ...,
\mathbf{n_R}^*).$
%\end{equation}
%where no user can benefit by deviating his/her strategy given all
%other users do not change.

The NE of this game represents the stable network state of the
interaction among all users.
%First we would like to know whether there exist such
%stable network states, and whether a dynamic interaction process
%will converge to a NE. Furthermore,
The network performance at a NE is described by the loss of
efficiency, defined as:
\begin{equation}
L_{eff} = B_{max}/B_{ne} \label{eqn:leff}
\end{equation}
where $B_{ne}$ is the total benefit of all users when the network
is at a NE, and $B_{max}$ is the maximum benefit. The worst
efficiency loss is also known as the \textit{price of anarchy}
\cite{koutsoupias99worstcase}.

\bigskip
\noindent \textbf{Remarks.} It is not necessarily true that the
throughput $G_r(\mathbf{n}_r)$
%defined in (\ref{eqn:throughput})
is an increasing function of $\mathbf{n}_r$. For example, in the
network shown in Figure \ref{fig:not.in}, user $r$ has three
paths: $p_1, p_2$ and $p_3$. $p_1$ contains two links $j_1$ and
$j_2$ with capacity $C$. $p_2$ contains link $j_1$ and $p_2$
contains link $j_2$. According to the simple rate allocation
mechanism introduced before, if $\mathbf{n}_r=(0,1,1)$, then
$G_r(\mathbf{n}_r)=2C$. However, if user $r$ increases its number
of connections on path $p_1$ from zero to one, then
$G_r(\mathbf{n}_r)=3C/2$. Thus, $G_r(\mathbf{n}_r)$ is a
decreasing function of $n_{p1}$.
%If for any user $r$,
%$G_r(\mathbf{n}_r)$ is an increasing function of $\mathbf{n}_r$,
%then there always exists a NE at extreme point
%$(\mathbf{n}_1^{max}, ..., \mathbf{n}_R^{max})$ in the strategy
%space where all users open their maximum allowable number of
%connections. However, as shown in Figure \ref{fig:not.in}, this is
%not necessarily true.




\begin{figure}[htb!]
    \begin{center}
%%\includegraphics[width=0.5\textwidth, height=0.25\textheight]{./network1_2.eps}
\includegraphics[scale=0.3]{./eps_BT_research/network7.eps}\\
        \caption{A case where the throughput of user $r$ is not increasing in $\mathbf{n}_r$.}
        \label{fig:not.in}
     \end{center}
\end{figure}

One interesting special case is that a user can only choose either
zero or one connection on a given available path. That is,
(\ref{eqn:usr.ss}) can be described as $n_{r_p}\in \{0, 1\},
\forall r_p \in P_r$. In this case, each user only has finite
number of strategies. This variant of the game is a finite game.
According to \cite{nash50}, this game admits a mixed strategy NE.
This NE is related to randomly choosing of connections to other
peers in BitTorrent applications \cite{bt}. This is an interesting
future research topic.
%In addition, as
%shown in Section \ref{sec:topo}, we incorporate Tit-for-Tat
%strategy into this variant of game to study the unstructured file
%sharing from a different angle.

%In the following section, we illustrate the possible Nash
%equilibrium on some simple network topology.


%\section{Resource Constrained Users}
%\noindent \textbf{Grid Network}\\

%\parskip=-10pt

\subsection{Existence of Multiple Nash Equilibria in Grid Network}
\label{sec:multiple}

%\parskip=0pt

In this section, we use a simple example to illustrate the
unstructured file sharing game and possible NEs. The network
topology in this example is a so called grid network introduced in
\cite{bonald01impact}, shown in Figure \ref{fig:fish}.(a). A
possible instance of this grid network is called ``fish" network,
shown in Figure \ref{fig:fish}.(b).

\begin{figure}[htb]
% FIRST TWO FIGS
\centerline{
    \begin{minipage}{1.6in}
     \begin{center}
        \setlength{\epsfxsize}{1.6in}
        \epsffile{eps_BT_research/grid-network-2.eps}\\
       %\newline{\small (a)}
       {\small (a)}
      \end{center}
    \end{minipage}
    \begin{minipage}{1.6in}
     \begin{center}
        \setlength{\epsfxsize}{1.1in}
        \epsffile{eps_BT_research/fish.eps}\\
        %\newline{\small (b)}
        {\small (b)}
      \end{center}
    \end{minipage}
}\caption{(a) is a grid network where
         squares represent links.
        (b) is an instance of (a).
        $A\to E\to B$ and $C\to D\to A$ correspond to route 1 and 2
        in (a).
        $D\to A\to E, E\to F\to D, C\to F\to B$
        correspond to routes $3,4,5$. }
        \label{fig:fish}
\end{figure}


%\dubfigsingle{epidemic_comp_delay.eps}{Average delay for different
%$N$}{eps_BT_research/fish.eps}{CDF of delay with $N=160$}{Delay
%under epidemic routing}{epidemic_delay}


A closed form rate allocation based on the $\alpha$-bandwidth
sharing mechanism for such a grid network is given in
\cite{bonald01impact}. Specifically, if there are $K$ horizontal
routes and $L$ vertical routes, then the total throughput on
horizontal path $p$ is given by
\begin{equation}
n_p y_p = \frac{ (\sum_{k=1}^K \frac{1}{RTT_k}
n_k^\alpha)^{1/\alpha} }{ (\sum_{k=1}^K \frac{1}{RTT_k}
n_k^\alpha)^{1/\alpha} + (\sum_{l=1}^L \frac{1}{RTT_l}
n_l^\alpha)^{1/\alpha}} \label{eqn:grid.rate}
\end{equation}
where $n_p$ denotes the number of flows on horizontal path $p$.
%$RTT_p$ denotes the round trip time of path $p$.
$y_p$ is the throughput of a single flow on path $p$.

In the following, we discuss two variants of the game by
considering two users playing the game on the grid network. User
$1$ uses route $1$ and user $2$ uses route $2$. Suppose $\alpha=2$
in (\ref{eqn:grid.rate}), which corresponds to TCP. Suppose that
all vertical and horizontal routes have RTT of $50$ms, and there
are $10$ background flows on all vertical routes.

\bigskip
\noindent \textbf{Benefit includes throughput only.} When both
users are only concerned with total throughput and have no
resource limitations, we have identified the following case
% where throughput $G$
%is neither a concave nor convex function of the number of
%connections for an unstructured file sharing game on the grid
%network.  This suggests that results on the existence of NE cannot
%be applied here because these results require the concavity of the
%utility function \cite{basar98game,rosen65}. We find that
where there is a unique NE, at which both players open their
maximal allowable number of connections.
%See \cite{zhang06tcpgame_general} for details.

%\begin{comment}
%For example,
%consider (\ref{eqn:grid.rate}) with $\alpha=2$
%corresponding to TCP.
%%Suppose that all vertical
%%and horizontal routes have RTT of $50$ms, all vertical routes have
%%$10$ flows.
There are two users. User $1$ uses the upper horizontal route and
user $2$ uses the lower horizontal route. Suppose that user $2$
opens $100$ connections. In Figure \ref{fig:thpt.p1.100}, we plot
the throughput of user 1 as a function of its number of
connections on its single available path.
% (route $1$ in Figure
%\ref{fig:grid}).
We find that the throughput of user $1$ is neither a concave nor
convex function of its number of connections on its single
available path.  This suggests that the current results on the
existence of Nash equilibrium cannot be applied here because these
results require the concavity of the utility function
\cite{basar98game}\cite{rosen65}.

However, note that the throughput of user $1$ is an increasing
function of $n_1$, which can be verified by checking its
first-order derivative. Similarly, we can also show that
%given
%user $1$ has a certain number of connections on its single
%available path, then
user $2$'s throughput is also an increasing function of its number
of connections. Therefore, if both users play the unstructured
file sharing game, there is a unique NE. Furthermore, at the NE
both players opens their maximal allowable number of connections.
%\end{comment}

%\begin{comment}
\begin{figure}[htb]
    \begin{center}
%%\includegraphics[width=0.5\textwidth, height=0.25\textheight]{./network1_2.eps}
\includegraphics[scale=0.25]{./eps_BT_research/player1_thpt_100.eps}\\
        \caption{Total data rate $G$ of user $1$ as a function
        of the number of connections on its path, when user 2 has 100 connections.}
        \label{fig:thpt.p1.100}
     \end{center}
\end{figure}
%\end{comment}

\bigskip
\noindent \textbf{Benefit includes both throughput and cost.} In
this variant of the game, not only is that $B_r$ neither a concave
nor a convex function of its number of connections $\mathbf{n}_r$,
but $B_r$ is not always increasing in $\mathbf{n}_r$.
%See \cite{zhang06tcpgame_general} for details.

%\begin{comment}
 For example,
suppose $\beta=0.0005$ in the cost function $\Phi(\mathbf{n}_r)$.
%(\ref{eqn:cost1}).
We plot in Figures \ref{fig:ben.p1.50} and \ref{fig:ben.p1.100}
the benefit $B$ of user $1$ as a function of its number of
connections on its single available route, given that the number
of connections of user $2$ is $50$ and $100$ respectively.
%Note
%that the benefit of player $1$ is neither concave nor convex
%function of the number of connections. In addition,
Note that, depending on the number of connections opened by user
$2$, the benefit of user $1$ can be either an increasing or a
decreasing function of $\mathbf{n}_1$.

\begin{figure}[htb!]
% FIRST TWO FIGS
\centerline{
        \begin{minipage}{1.6in}
     \begin{center}
        \setlength{\epsfxsize}{1.6in}
        \epsffile{eps_BT_research/player1_benefit_50.eps}\\
       {}
        \caption{Benefit of user $1$ as a function of the number of connections
         when user $2$ has $50$ connection. }
        \label{fig:ben.p1.50}
      \end{center}
    \end{minipage}
   \begin{minipage}{1.6in}
     \begin{center}
        \setlength{\epsfxsize}{1.6in}
        \epsffile{eps_BT_research/player1_benefit_100.eps}\\
       {}
        \caption{Benefit of user $1$ as a function of the number of connections
        when user $2$ has $100$ connection. }
        \label{fig:ben.p1.100}
      \end{center}
    \end{minipage}
    }
\end{figure}

%\end{comment}


%Consider that both players have the same $\beta=0.0005$.
We define the best response $\mathbf{n}_r^*$ of player $r$ as the
solution of $r$'s optimization problem given fixed strategies of
all other players $\mathbf{n}_{-r}$. In Figure \ref{fig:p1.p2.br},
we plot the best response curves of both players. Note that there
are three intersecting points. An intersecting point is a NE
because at that point, each user's response is the best response
to the other user's strategy. Thus, there are three NE in this
game. For comparison, in the single link TCP connection game
\cite{zhang05tcpgame_icnp}, there is only one unique NE when the
cost is proportional to the number of connections.

\begin{figure}[htb!]
    \begin{center}
%%\includegraphics[width=0.5\textwidth, height=0.25\textheight]{./network1_2.eps}
\includegraphics[scale=0.25]{./eps_BT_research/player1_player2_br.eps}\\
        \caption{Best response curves of both player 1 and player 2.}
        \label{fig:p1.p2.br}
     \end{center}
%created by matlab script tcp_grid.m in
%C:\honggang\Toshiba_D\research\proposal\BT_research
\end{figure}

It is also interesting to note that these two players do not share
any common link (Figure \ref{fig:fish}), so, their interaction
arises because they share links with other common sessions.

This simple example indicates that the interaction among multiple
users on a general network topology can be much more complex than
the single link TCP connection game. The existence and uniqueness
of Nash equilibrium can depend on network topologies and the
utility functions adopted by users.
%, as demonstrated by the
%previous examples.



% Figure \cite{}
%shows the throughput obtained by this user as a function of the
%number of flows he/she puts on this route. Note that this
%throughput is neither a concave nor convex function of the number
%of connections on this path. However, this throughput is a
%strictly increasing function of the number of connections. Thus,
%in the game on this grid topology, a unique Nash equilibrium is
%$\mathbf{n}=(n^1_{max},..., n^R_{max})$ where all users open their
%maximum allowable number of connections.




\begin{comment}
\begin{figure*}[htb!]
% FIRST TWO FIGS
\centerline{
    \begin{minipage}{2.2in}
     \begin{center}
        \setlength{\epsfxsize}{2in}
        \epsffile{eps_BT_research/player1_benefit_1.eps}\\
       {}
        \caption{Benefit of user 1 as a function of the number of connections opened by
        user 1 when user 2 has 1 connection. }
        \label{fig:ben.p1.1}
      \end{center}
    \end{minipage}
    \begin{minipage}{2.2in}
     \begin{center}
        \setlength{\epsfxsize}{2in}
        \epsffile{eps_BT_research/player1_benefit_10.eps}\\
       {}
        \caption{Benefit of user 1 as a function of the number of connections opened by
        user 1 when user 2 has 10 connection. }
        \label{fig:ben.p1.10}
      \end{center}
    \end{minipage}
        \begin{minipage}{2.2in}
     \begin{center}
        \setlength{\epsfxsize}{2in}
        \epsffile{eps_BT_research/player1_benefit_50.eps}\\
       {}
        \caption{Benefit of user 1 as a function of the number of connections opened by
        user 1 when user 2 has 50 connection. }
        \label{fig:ben.p1.50}
      \end{center}
    \end{minipage}
    }
\centerline{
    \begin{minipage}{2.2in}
     \begin{center}
        \setlength{\epsfxsize}{2in}
        \epsffile{eps_BT_research/player1_benefit_100.eps}\\
       {}
        \caption{Benefit of user 1 as a function of the number of connections opened by
        user 1 when user 2 has 100 connection. }
        \label{fig:ben.p1.100}
      \end{center}
    \end{minipage}
    \begin{minipage}{2.2in}
     \begin{center}
        \setlength{\epsfxsize}{2in}
        \epsffile{eps_BT_research/player1_benefit_150.eps}\\
       {}
        \caption{Benefit of user 1 as a function of the number of connections opened by
        user 1 when user 2 has 150 connection. }
        \label{fig:ben.p1.150}
      \end{center}
    \end{minipage}
        \begin{minipage}{2.2in}
     \begin{center}
        \setlength{\epsfxsize}{2in}
        \epsffile{eps_BT_research/player1_benefit_200.eps}\\
       {}
        \caption{Benefit of user 1 as a function of the number of connections opened by
        user 1 when user 2 has 200 connection. }
        \label{fig:ben.p1.200}
      \end{center}
    \end{minipage}
}
\end{figure*}

\end{comment}






\begin{comment}

\begin{figure}[htb]
% FIRST TWO FIGS
\centerline{
    \begin{minipage}{3in}
     \begin{center}
        \setlength{\epsfxsize}{3in}
        \epsffile{eps_BT_research/concave3.eps}\\
       {}
        \caption{Total data rate $G^i$ of user $i$ as a function
        of the number of connections on both paths. Side view. }
        \label{fig:cost.total.n100}
      \end{center}
    \end{minipage}
    }
\centerline{
    \begin{minipage}{3in}
      \begin{center}
        \setlength{\epsfxsize}{3in}
        \epsffile{eps_BT_research/concave3_2.eps}\\
       {}
        \caption{Total data rate $G^i$ of user $i$ as a function
        of the number of connections on both paths. Top view. }
        \label{fig:cost1.n100}
      \end{center}
    \end{minipage}
}
\end{figure}


\begin{figure}[htb]
% FIRST TWO FIGS
\centerline{
    \begin{minipage}{2.5in}
     \begin{center}
        \setlength{\epsfxsize}{2.5in}
        \epsffile{eps_BT_research/grid_1.eps}\\
       {}
        \caption{Total data rate $G^i$ of user $i$ as a function
        of the number of connections on both paths. Side view. }
        \label{fig:cost.total.n100}
      \end{center}
    \end{minipage}
    }
\centerline{
    \begin{minipage}{3in}
      \begin{center}
        \setlength{\epsfxsize}{3in}
        \epsffile{eps_BT_research/grid_2.eps}\\
       {}
        \caption{Total data rate $G^i$ of user $i$ as a function
        of the number of connections on both paths. Top view. }
        \label{fig:cost1.n100}
      \end{center}
    \end{minipage}
}
\end{figure}
\end{comment}

%However, if we consider the payoff including the throughput and the cost
%as a function of $\mathbf{n^i}$, then we might end up with a function
%neither increasing nor decreasing, besides neither concave nor convex.
%For example, if we consider payoff as $B^i=G^i - 0.0004 n^i_p$,
%we plot the payoff of this user in Figure \ref{}.

In the following, we focus on two special networks: a parallel
link network and a star network. Both can be used to model
peer-to-peer networks.

%\parskip=-10pt

%\input{tcpgame-general-parallel}

\section{Parallel Link Network}\label{sec:para}

%\parskip=0pt

In this section, we investigate an unstructured file sharing game
on a parallel-link network where all users share a common source
and a common destination node interconnected by a number of
parallel links.
%This type of network is not only practically interesting itself,
%but also can be thought of as some special case of a star network
%addressed in the next section.
Parallel-link networks can be used as simple models for
unstructured file sharing. For example, in eDonkey networks
\cite{p2p}, a peer can download a file from multiple other peers
providing this file. There are possibly many peers simultaneously
downloading the same file, and they can be thought of as
associated with a common destination node. Each of the
file-providing peers can be thought of as a ``link" or ``path"
connecting the common destination node with a common super virtual
file-providing source node. Those downloading peers compete for
these parallel links/paths for bandwidth. This scenario can be
approximated by a parallel link network.



%Parallel-link networks can represent many practical situations.
%For example \cite{korilis97achieving}, in broadband networks,
%bandwidth is separated among different virtual paths, resulting in
%a network of parallel and noninterfering ``links" between source
%and destination pairs.
%%Another scenario can be that
%%some organization can buy service from several different network
%%providers, and then split its total traffic over the various
%%network facilities, each of which can be represented as a link in
%%the parallel link topology.
%In addition, this parallel link
%network topology can also be used to model balancing the load
%among several different resources \cite{roughgarden01stackelberg}.
In this section, we first show the existence of stable network
states (NEs) on a parallel-link network. We then present the
results on the efficiency loss of NE and the stability of NE in
the best-response dynamics.

%\parskip=-10pt

\subsection{Nash equilibrium}

%\parskip=0pt

Suppose that there are $L$ links and $R$ users. By an abuse of
notation, we will use $L$ and $R$ to denote the set of links and
the set of users respectively.
%The utility function of user $r$ is
%$U_r$. The cost incurred to user $r$ is a convex and
%non-decreasing function of the number of connections opened by
%user $r$.
An example of a parallel link network is shown in Figure
\ref{fig:network4}.
%In this parallel link network,
The throughput $G_{rj}$ obtained by user $r$ on link $j$ is given
by the simple rate allocation mechanism introduced in the previous
section:
 $G_{rj}(n_{rj})=C_j n_{rj}/RTT_{rj}/(\sum_{k=1}^R
n_{kj}/RTT_{kj})$, where $RTT_{rj}$ is the Round Trip Time of user
$r$ on link/path $j$, $C_j$ is the capacity of link $j$, and
$n_{rj}$ is the number of connections of user $r$ on link $j$. The
strategy of user $r$ is a vector of the number of connections on
its available paths or links: $\mathbf{n}_r=(n_{r1},...,n_{rL})$
and $n_{rj}\in (0, n^{max}_r], \forall j\in L$. $n^{max}_r$ is the
maximum allowable number of connections for user $r$. Note that
this game is a continuous kernel game \cite{basar98game} as we
assume that a user's strategy is a real-valued vector.
\begin{figure}[htb!]
    \begin{center}
%%\includegraphics[width=0.5\textwidth, height=0.25\textheight]{./network1_2.eps}
\includegraphics[scale=0.3]{./eps_BT_research/network4_2.eps}
\caption{A parallel-link network topology.} \label{fig:network4}
     \end{center}
\end{figure}

In this section, we only consider the case where
$U_r(\mathbf{n}_r)=G_r(\mathbf{n}_r)$. The benefit or payoff
obtained by user $r$ is: $B_r(\mathbf{n}_r) =
G_r(\mathbf{n}_r)-\Phi_{r}(\mathbf{n}_r)$.




We consider two scenarios: an unconstrained game and a constrained
game. In an unconstrained game, there is no upper limit on the
total number of connections a user can open. In a constrained
game, each user must choose a certain total number of
connections\footnote{This is motivated by BitTorrent \cite{bt}
where each peer always has $5$ active connections open to $5$
different other peers.}. We have shown the existence of a unique
NE in both constrained and unconstrained games.

%More results for the constrained game can be found in
%\cite{zhang06tcpgame_general}. In the following we only present
%results for the unconstrained game.


\subsection{Unconstrained Game}
In an unconstrained game,
%users can open arbitrarily large number of connections.
users essentially play an independent game on each distinct
path/link. Since a NE exists and is unique on a single link game
\cite{zhang05tcpgame_icnp}, we know that a NE also exists and is
unique on this parallel link network. This is summarized in the
following theorem.

\begin{theorem}
There exists a unique interior-point NE in an unstructured file
sharing game on a parallel link network.
%when all users have either Type 1 or Type 2 costs.
\end{theorem}

\bigskip
\noindent \textbf{Social Benefit at Nash equilibrium.} As shown in
\cite{zhang05tcpgame_icnp}, a single bottleneck link TCP
connection game admits a symmetric NE when users have the same
Round Trip Times (RTT) and their benefit function includes
throughput and a cost proportional to the number of connections.
This result can be extended to our uncontrained game. That is,
when all users have the same RTTs, the unique NE is symmetric, in
the sense that all users have the same number of connections at
the NE.

Solving the optimization problem for a user $r$, we can get the
vector of connections of user $r$ at the symmetric NE as:
$$
n^*_{rj} = (R-1) C_j / (R^2 \beta).$$ Then, user $r$'s benefit at
the NE is $$ B_r^*=\sum_{j=1}^L C_j/R - \sum_{j=1}^L
(R-1)C_j/R^2.$$ Therefore, the total social benefit of the NE is
$$B_{ne}=\sum_{j=1}^{L} C_j/R.$$ Note that $B_{ne}$ is not related
to the cost of users. It is simply a function of the total network
capacity and the number of users. As the number of users
increases, the total social benefit of the NE goes to zero.

\bigskip
\noindent \textbf{Reaction functions.} The reaction function of a
user $r$ is defined as the best response of user $r$ as a function
(if it exists) of the total number of connections of all other
users. A response of user $r$ is
%a vector
%of numbers of connections on all links, that is,
$\mathbf{n_r}=(n_{r1},n_{r2},...,n_{rL})$. Since in an
unconstrained game users essentially play an independent game on
each individual link, we can solve for a user's best response on
each link separately. Specifically, for any link $j$, we have
\begin{equation}
\bar{n}_{rj}=\mbox{argmax}_{n_{rj}\in (0, \infty)}
B_{rj}(\sum_{k\ne r}^R n_{kj}).
\end{equation}
For convenience,
let $n_{-rj}$ denote $\sum_{k\ne r}^R n_{kj}$. It is easy to show
that
\begin{equation}
\bar{n}_{rj}= f(n_{-rj}) = -n_{-rj} + \sqrt{C_j n_{-rj}/\beta}.
\label{eqn:react}
\end{equation}

$\bar{n}_{rj}$ is a continuous function of $n_{-rj}$. We note that
in order to guarantee that the best response of user $r$ is an
interior point of its strategy space, we must have
\begin{equation}
\bar{n}_{rj} > 0 \quad \mbox{or} \quad n_{-rj} < C_j /\beta.
\end{equation}
%\nonumber
%\end{equation}

%Please see\cite{zhang06tcpgame_general} for more details.

As shown in Section \ref{sec:multiple}, we can use reaction
functions to identify NEs by checking the intersecting point(s) of
the reaction function (best response) curves of all players. We
can also use reaction functions to investigate the best-response
dynamics of the game playing process, as discussed later.

\bigskip
\noindent \textbf{Stability of NE in Best-response Dynamics}.
Suppose that users interact with each other using best-response in
a discrete time process, a so called \textit{best-response
dynamics} \cite{basar98game}\cite{zhang05tcpgame_icnp}. This
process proceeds in discrete time steps or rounds, and only one
randomly chosen user makes a move at each round.
%At each round,
%one randomly chosen user makes a move.
Whenever a user makes a move, it calculates its best response to
other users' numbers of connections which are determined in
previous steps. That is, the user who makes a move solves its
optimization problem to maximize its benefit. If all users'
strategies converge to or stabilize at some point $\mathbf{n}_s$
as time goes to infinity, then $\mathbf{n}_s$ is a NE, and it is
\textit{globally stable}. Regarding an unstructured file sharing
game on a parallel link network, we have the following stability
result.
%Please refer to \cite{zhang06tcpgame_general} for detailed proof.


\begin{theorem}
The unique NE is globally stable in the two-player version of the
unstructured file sharing game on parallel link network when both
players use best-response to play the game.
\end{theorem}
\begin{proof}
We want to show that the best response of a user is a concave
function of the other player's number of connections. In the
unconstrained game, users actually play independent games on
different links. For a given user $r$, the best response function
or reaction function on link $j$ is given by (\ref{eqn:react}),
and re-stated as follows:
$$
\bar{n}_{r,j}=-n_{-r,j} + \sqrt{C_j n_{-r,j}/\beta},
$$
where $n_{-r,j}$ is the number of connections of all other users.
It can be shown that
$$
\partial^2 \bar{n}_{r,j}/\partial n_{-r,j}^2 = (-1/4)\sqrt{C_j/\beta} \cdot n_{-r,j}^{-3/2} \le 0.
$$
Thus, the reaction function of user $r$ is a concave function of
number of connections of other users. Then, from
\cite{zhang05tcpgame_tech}, we know that in a two-player version
of the game, Nash equilibrium is globally stable.
\end{proof}




%\cite{zhang06tcpgame_general} that the Nash equilibrium in the
%two-player version of unconstrained game is globally stable if
%players use best-response to play the game.



%If all links have the same capacity $C$,
%then social benefit of Nash equilibrium is as follows.
%\begin{eqnarray}
%B^* &=& \sum_{i=1}^n B^{i*} \nonumber \\
%    &=& n (m C/n - m (n-1) C /n) \nonumber \\
%    &=& mC /n
%\end{eqnarray}


%\parskip=-10pt
\bigskip
\noindent \textbf{Efficiency loss of Nash equilibrium}

%\parskip=0pt

%\subsubsection{Unconstrained game}
First note that the maximal system benefit is the solution of a
straightforward optimization problem. The system benefit can be
represented as:
\begin{equation}
B = \sum_{r=1}^R B_r = \sum_{r=1}^R \sum_{j=1}^L G_{rj} -
  \beta \sum_{r=1}^R  \sum_{j=1}^L n_{rj}. \label{eqn:B.pa}
\end{equation}
We find that the maximal value of $B$
%(\ref{eqn:B.pa})
is
\begin{equation}
B_{max} = \sum_{j=1}^L C_j - \beta N_{min}. \label{eqn:B.pa.1}
\end{equation}

Consider a homogeneous network where all links have the same
capacity. Then we have $B_{max}= L C - \beta L$, as we need at
least one connection for each link in order to get the bandwidth
of each link. The efficiency loss of a NE is given by
\begin{equation}
L_{eff}= \frac{B_{max}}{B_{ne}}= \frac{L C-\beta L}{LC/R}.
\end{equation}
This result essentially suggests that the efficiency loss of the
unique NE is bounded.
%If $L,R$ are fixed, and let $C\to \infty$, then $L_{eff} \to R$.
However, if $L,C$ are fixed, and let $R\to \infty$, then $L_{eff}
\to \infty$. This suggests that the system performance at NE can
degrade arbitrarily if the number of users becomes large.

%\parskip=-10pt
\bigskip
\noindent \textbf{Socially Responsible Users}

%\parskip=0pt

Note that we can think of users as data senders in the game
discussed above.
%The links in the parallel network are bottleneck
%links.
Let the packet loss rate associated with each link/path $j$ be
$p_j$. Suppose that the packet sending rate of a TCP connection of
user $r$ on path/link $j$ is $T_{rj}$. The throughput of this
connection is given by $G_{rj}=T_{rj}(1-p_j)$. Not all packets
coming to bottleneck link $j$ are delivered. The network resources
before link $j$ are partially wasted because that they carry data
at a higher rate than the actual delivery rate of link $j$.
Therefore we can think of this extra traffic as a cost to the
network and that is proportional to the packet sending rate
$T_{rj}$. A user is considered as socially responsible if his/her
benefit function includes this cost term. That is, we have
$B_r(\mathbf{n}_r)= G_{rj}-\gamma \sum_{i=1}^L n_{rj} T_{rj}$,
where $\gamma\in (0,1)$. Based on
%the result on a single link TCP
%connection game
\cite{zhang05tcpgame_icnp}, we can show that there exists a pure
strategy unique NE because users actually play a game on each link
independently from other links. It also follows that the loss of
efficiency of the NE is bounded as the unique NE is an interior
point in the strategy space.
%See \cite{zhang06tcpgame_general} for more details.
%\footnote{
Note that the definition of loss of efficiency in unstructured
file sharing game is different from that of the single bottleneck
link TCP connection game. The latter is defined as the ratio of
total sending rate from all users at NE over the minimum total
sending rate. The latter is the efficiency loss from network's
point of view, whereas the former is from user's point of view.


\subsection{Constrained Game}

Consider another model where the total number of connections that
are allowed to open by a user is fixed. Formally, for any user
$r$, we have $\sum_{j}^L x_{rj}= n_r$, where $n_r$ is the required
total number of connections.

%Then, a user $r$ has the following constrained optimization
%problem.
%\begin{eqnarray}
%\mbox{maximize} & B_r(\mathbf{n}_r)    \\
%\mbox{subject to} & \sum_{j}^L n_{rj}= n_r \\
%& \mathbf{n}_r = (n_{r1},n_{r2},...,n_{rL})\nonumber \\
%& n_{rj} \in (0,\infty) \nonumber
%\end{eqnarray}

We refer to this game as a \textit{constrained game}. As
summarized in the following theorem, this game admits a unique
symmetric Nash equilibrium.
%Please see \cite{zhang06tcpgame_general} for the
%proof of this theorem.

Please see Appendix \ref{appd:con} for the proof of this theorem.

\begin{theorem}
There exists a unique interior-point symmetric Nash equilibrium in
a constrained unstructured file sharing game in parallel-link
network. \label{them:con}
\end{theorem}

\bigskip
\noindent \textbf{Remarks.} It can be true that there are
\textit{asymmetric} NE. For example, suppose that there are two
users and two links with the same capacity, and each user is
constrained to use two and only two connections. Then one NE is
that user 1 opens its two connections on link 1 and user 2 opens
its two connections on link 2, or a NE could be that user 1 opens
its two connections on link 2, and user 2 opens its two
connections on link 1.

\bigskip
\noindent \textbf{An Illustrative Example for the existence and
stability of NE.} We use a simple example to illustrate the Nash
equilibrium proved in Theorem \ref{them:con} . There are three
users: $A$, $B$, and $C$. There are two paths (or two links) in a
parallel link topology. Suppose that the capacity of link 1 is
$C_1=25$Mbps and the capacity of link 2 is $C_2=100$Mbps. Suppose
that each user has to open $20$ connections. As proved in Theorem
\ref{them:con}, at Nash equilibrium, each user will open $4$ and
$16$ connections on link 1 and 2 respectively, because
$n^*_1/n^*_2=C_1/C_2=1/4$. That is, at Nash equilibrium we have, $
\mathbf{n^*_A}=\mathbf{n^*_B}=\mathbf{n^*_C}=(4,16). $

Suppose users interact with each other using \textit{best-response
dynamics} \cite{basar98game}\cite{zhang05tcpgame_icnp}. If all
users' strategies converge to or stabilize at some points as time
goes by, then the stablized numbers of connections are the Nash
equilibrium strategies for all users. As shown in Figure
\ref{fig:parallel_1_conn}, the best-response dynamics indeed
converges to a stable point which corresponds to the Nash
equilibrium obtained from the previous analysis.


\begin{figure}[htb!]
% FIRST TWO FIGS
\centerline{
    \begin{minipage}{1.6in}
     \begin{center}
        \setlength{\epsfxsize}{1.6in}
       \epsffile{eps_BT_research/parallel_1_benefit.eps}
          \end{center}
    \end{minipage}
    \begin{minipage}{1.6in}
      \begin{center}
        \setlength{\epsfxsize}{1.6in}
        \epsffile{eps_BT_research/parallel_1_conn.eps}
      \end{center}
    \end{minipage}
    }
\caption{An example of the best-response dynamics on two parallel
links. This dynamic process converges to Nash equilibrium. The
left figure shows the benefit of three users. The right figure
shows the numbers of connections.} \label{fig:parallel_1_conn}
\end{figure}

%\bigskip
\noindent \textbf{Loss of Efficiency.}
 Given the constraint
that the total number of connections of user $r$ should be equal
to $n_r$, the maximal value of (\ref{eqn:B.pa}) is given by $
B_{max}= \sum_{j=1}^L C_j - \beta\sum_{r=1}^R n_r$. The system
optimal performance is exactly the same as the system performance
at Nash equilibrium. Then, the Nash equilibrium has no efficiency
loss, that is, $L_{eff}=1.$

%\noindent \textbf{System performance of Nash equilibrium when users have
%cost type 1.}
%$$
%\sum_r B_r = \sum_{k=1}^L C_k - \beta \sum_r n_r.
%$$

%\subsection{Proof of Theorem \ref{them:con}}
%\label{appnd:constrained}





%\bigskip
% (... TO BE FILLED, stability of NE in constrained game...)
%\bigskip


%It is interesting to note that Figure \ref{fig:parallel_1_benefit}
%and Figure \ref{fig:parallel_1_conn} actually suggest that the
%unique Nash equilibrium is globally stable in the best-response
%dynamics.
%\end{comment}

\begin{comment}
Depending the values of the number of users $m$ and the number of
links $n$, we can have the following two cases.

\noindent \textbf{Case 1, more users than links.} If $R>L$, then
$X_{min}=m$, and we have $ B_{max}=mC-m.$ In this case, the
efficiency loss of Nash equilibrium is
\begin{equation}
L_{eff}= B_{max}/B_{ne}= \frac{mC-m}{mC- m(n-1)C /n} = n(1-1/C).
\end{equation}
Asymptotically, if $n \to \infty$, $L_{eff}\to \infty$. On the
other hand, if $m,n$ are fixed, and we let $C\to \infty$, then
$L_{eff}\to n$.

\noindent \textbf{Case 2, more links than users.} If $m\le n$,
$X_{min}=n$, then $B_{max}= mC - n.$ Efficiency loss of Nash
equilibrium is
\begin{equation}
L_{eff}= B_{max}/B_{ne}= (mC-n)/(mC/n)= 1- n/(mC).
\end{equation}
If $m,n$ are fixed, and let $C\to \infty$, then $L_{eff} \to 1$.
If $m,C$ are fixed, and let $n\to \infty$, then $L_{eff} \to
-\infty$.

\end{comment}

%\input{tcpgame-general-star}

%\parskip=-5pt

\section{Star Network}\label{sec:star}

%\parskip=0pt

%We consider a star network in this section.
In this section, we use a star network to approximately model a
peer-to-peer file sharing overlay network, and investigate the
unstructured file sharing game on such a star network. Figure
\ref{fig:network3} presents one such example.

In the star network, we assume that a user has two asymmetric
access links to the Internet: one downstream link and one upstream
link. This assumption is supported in a measurement study in
\cite{saroiu02measurement}, where it is found that most users in
current peer-to-peer networks use cable modem or ADSL to get
connected to the Internet. Usually the downstream link has higher
capacity than the upstream link \cite{saroiu02measurement}.

A user $r$ uses its downstream link to get data from other peers.
The downstream link of user $r$ is a ``private" link in the sense
that this link is only used by user $r$ itself. On the other hand,
the upstream link of user $r$ is shared by all other peers or
users who are downloading files from user $r$. We can think of the
upstream link of user $r$ as a ``public" link from the point of
view of user $r$.

In addition, similar to \cite{qiu04bt}\cite{piccolo04p2p}, we
assume that in a peer-to-peer file sharing network, bottlenecks
can occur at access links, not in the core Internet. This
assumption is a reasonable approximation of the current
peer-to-peer file sharing networks such as Gnutella and
BitTorrent, where usually the data throughput is limited by the
``last mile" (cable or ADSL or modem) of a connection. Thus, in
the star network shown in Figure \ref{fig:network3}, the Internet
cloud can be represented simply as a central node.


In the following,
%we first describe the characteristics of this
%star network and give justifications based on previous studies.
%measurement studies and previous research on peer-to-peer
%networks.
we first prove the existence of NE in unstructured file sharing
game on a star network. We then use examples to illustrate the
best response dynamics of this game playing process, and finally
we present our results on the loss of efficiency of NE.

\begin{comment}
\begin{figure}[htb!]
    \begin{center}
    \includegraphics[width=0.2\textwidth, height=0.1\textheight]{./eps_BT_research/network3_5_2_new.eps}
    %\includegraphics[scale=0.3]{network3_5_2.eps}
\caption{An example of star network topology.}
\label{fig:network3}
    \end{center}
\end{figure}
\end{comment}

\begin{figure}[htb!]
% FIRST TWO FIGS
\centerline{
    \begin{minipage}{1.8in}
     \begin{center}
        \setlength{\epsfxsize}{1.2in}
       \epsffile{eps_BT_research/network3_5_2_new.eps}
          \end{center}
\caption{An example of star network.} \label{fig:network3}
    \end{minipage}
    \begin{minipage}{1.6in}
      \begin{center}
        \setlength{\epsfxsize}{1.6in}
        \epsffile{./eps_BT_research/network1.eps}
      \end{center}
\caption{A three node topology.} \label{fig:network1}
    \end{minipage}
    }
\end{figure}

%\parskip=-10pt

\subsection{Nash Equilibrium}

%\parskip=0pt

Recall that the benefit of user $r$ is given by
(\ref{eqn:benefit}).
%\begin{equation}
%B_r(\mathbf{n_r}) = U_r(G_r(\mathbf{n_r})) -
%\Phi_r(\mathbf{n_r}).\label{eqn:benefit.star}
%\end{equation}
In the following, we first present a lemma (Lemma \ref{lem:lem1})
and later use it to prove that a utility
function\footnote{$U_r(x)$ is assumed to be continuous,
nondecreasing, and concave.} $U_r(G_r(\mathbf{n_r}))$ is a
non-decreasing, continuous, and concave function of user $r$'s
number of connections $\mathbf{n_r}=(n_{r1}, ..., n_{rP_r})$,
where $P_r$ represents the set of available paths of user $r$ and
the number of paths as well. Since we assume that cost
$\Phi_r(\mathbf{n}_r)$ is an increasing and convex function of
$\mathbf{n}_r$, it then follows that the benefit $B_r$ is a
non-decreasing, continuous, and concave function of
$\mathbf{n_r}$.

Lemma \ref{lem:lem1} is introduced for the simple network in
Figure \ref{fig:network1}, where a user $r$ has two paths ($A\to C
\to D$ and $B\to C \to D$) to transfer data to destination node
$D$. Both paths share a common link $CD$.
%(++++ THis is needed for tech report)
%Link $AC$, $BC$, and $CD$ have
%capacities $C_1$, $C_2$, and $C_3$ respectively.
Suppose that the number of connections user $r$ opens on path
$A\to C \to D$ is $n_{p1}$, and on path $B\to C \to D$ is
$n_{p2}$. Then we have $\mathbf{n}_r=(n_{p1}, n_{p2})$.

We assume that link $CD$ is a private link of user $r$, i.e., no
other users use this link. This private link corresponds to the
downstream link of user $r$ in a star network. On the other hand,
links $AC$ and $BC$ are shared by user $r$ and other users. $AC$
and $BC$ correspond to two public links of user $r$ in a star
network.
%(++++ THis is needed for tech report)
%We use $n_{-r1}$ denote the number of connections opened on link
%$AC$ by users other than user $r$. Similarly, let $n_{-r2}$ denote
%the number of connections opened on link $BC$ by users other than
%user $r$.

\begin{comment}
\begin{figure}[htb!]
    \begin{center}
%\includegraphics[width=0.3\textwidth, height=0.15\textheight]{./network1.eps}
\includegraphics[scale=0.5]{./eps_BT_research/network1.eps}
\caption{A three-node topology.
%Link $AC$ has capacity $C_1$, and
%link $BC$ has capacity $C_2$, and link $CD$ has capacity $C_3$.
} \label{fig:network1}
     \end{center}
\end{figure}
\end{comment}

Recall that throughput $G_r$ obtained by user $r$ is a function of
$\mathbf{n}_r$. Lemma \ref{lem:lem1} shows that $G_r$ is a concave
function of $\mathbf{n}_r$.
%A detailed proof and an illustrative example are given in
%\cite{zhang06tcpgame_general}.


\begin{lemma}
Throughput $G_r$ of user $r$ in Figure \ref{fig:network1} is a
concave function of
%its vector of numbers of connections denoted
%as
$\mathbf{n}_r=(n_{p1}, n_{p2})$. \label{lem:lem1}
\end{lemma}

%\begin{comment}

\begin{proof}
The strategy vector of user $r$ is $\mathbf{n}_r=(n_{p1},
n_{p2})$. Let $z=\frac{n_{p1} C_1}{n_{p1} + n_{-r1}} +
\frac{n_{p2} C_2}{n_{p2}+n_{-r2}}$.

Then, the throughput obtained by user $r$ is
\begin{equation}
G(n_{p1},n_{p2}) = \left \{ \begin{array}{ll}
C_3 & \textrm{, if $z \ge C_3$} \\
z & \textrm{, if $z \le C_3$}
\end{array} \right.
\end{equation}

%\frac{C_3(x_{p1}+x_{p2})}{x_{p1}+x_{p2}+y_3} & \textrm{if $z<C_3$}

First, we note that this function is continuous and increasing.
Second, this function has two parts, with each part being a
concave function. Now we want to show that this function is a
concave function of $\mathbf{n}_r$ \textit{everywhere} in its
domain.

Take any two points $\mathbf{n^1}$ and $\mathbf{n^2}$. Without
loss of generality, we assume that $\mathbf{n^1}$ satisfies $z \le
C_3$ and that $\mathbf{n^2}$ satisfies $z \ge C_3$, as shown in
Figure \ref{fig:two.regions}. We would like to show that
$$
G(\delta \mathbf{n^1} + (1-\delta) \mathbf{n^2}) \ge \delta
G(\mathbf{n^1})+ (1-\delta) G(\mathbf{n^2}), \delta \in [0,1].
$$
%Let $D$ denote region $z \ge C_3$.

\begin{figure}[htb!]
    \begin{center}
    %%\includegraphics[width=0.5\textwidth, height=0.25\textheight]{./network1_2.eps}
    \includegraphics[scale=0.6]{./eps_BT_research/graph1.eps}
    \caption{The domain of $G_r$,
    the throughput of user r, can be divided into two regions.
    One region is $z\ge C_3$, and the other region is $z\le C_3$.}
    \label{fig:two.regions}
    \end{center}
\end{figure}



If we connect points $\mathbf{n^1}$ and $\mathbf{n^2}$ with a
line, then this line intersects with the boundary of region $z \ge
C_3$ at point $\mathbf{n^0}$. Then we have,
\begin{eqnarray}
G(\delta \mathbf{n^1} + (1-\delta) \mathbf{n^2}) &\ge &
G(\delta \mathbf{n^1} + (1-\delta) \mathbf{n^0}) \label{eqn:lec3} \\
&\ge & \delta G(\mathbf{n^1})+ (1-\delta) G(\mathbf{n^0}) \label{eqn:concave}\\
&=& \delta G(\mathbf{n^1})+ (1-\delta) G(\mathbf{n^2})
\label{eqn:contin}
\end{eqnarray}

(\ref{eqn:lec3}) is true because that $G(\mathbf{x})$ is an
increasing function of $\mathbf{x}$, and $\delta \mathbf{n^1} +
(1-\delta) \mathbf{n^2} \ge \delta \mathbf{n^1} + (1-\delta)
\mathbf{n^0}$. (\ref{eqn:concave}) is true because function $G$ is
a concave function in region $z \le C_3$. (\ref{eqn:contin}) is
true because function $G$ is a continuous function.

\end{proof}

%\end{comment}

%\begin{comment}

\bigskip
\noindent \textbf{An illustrative example.} In Figure
\ref{fig:network1}, suppose we choose $6$bps as capacities for
links $A\to C$ and $B\to C$ and $2$bps for link $C\to D$. User $r$
wants to open some number of connections on paths $A\to C\to D$
(path $1$) and $B\to C\to D$ (path $2$) to transfer data from $A$
and $B$ to destination node $D$.
%Consider user $r$ has the
%following benefit function
%\begin{equation}
%B_r(\mathbf{n_r}) = G_r(\mathbf{n_r}) -
%\Phi_r(\mathbf{n_r}).\label{eqn:benefit.star.simple}
%\end{equation}
%Suppose $\beta=0.005$ in user's benefit function
%(\ref{eqn:benefit.star.simple}).
The numbers of connections or sessions from other users on links
$AC$ and $BC$ are $100$. We vary the numbers of connections from
user $r$ on path $1$ and $2$, and then compute the throughput
received by user $r$.
% according to
%(\ref{eqn:benefit.star.simple}).
As shown in Figure \ref{fig:cost.total.n100}, we see that user
$r$'s throughput is indeed a concave function.


\begin{figure}[htb!]
% FIRST TWO FIGS
\centerline{
    \begin{minipage}{1.6in}
     \begin{center}
        \setlength{\epsfxsize}{1.6in}
        \epsffile{eps_BT_research/concave2.eps}\\
       {}
      \end{center}
    \end{minipage}
    \begin{minipage}{1.6in}
      \begin{center}
        \setlength{\epsfxsize}{1.6in}
        \epsffile{eps_BT_research/concave2_2.eps}\\
       {}
      \end{center}
    \end{minipage}
} \caption{Throughput $G_r$ of user $r$ as a function
        of the number of connections on both paths. The left figure is a side view. The
        right figure is a top view. }
        \label{fig:cost.total.n100}
%        \label{fig:cost1.n100}
\end{figure}

%\end{comment}

Consider the network in Figure \ref{fig:network2}, a generalized
version of the network in Figure \ref{fig:network1}. In Figure
\ref{fig:network2}, there are $M$ (multiple) paths along which
user $r$ can get data from the sender. All paths share a common
link $BA$. A strategy vector of user $r$ is $\mathbf{n}_r=(n_{r1},
n_{r2}, ..., n_{rM})$ with $M\ge 2$. We can extend the result in
Lemma \ref{lem:lem1} to show that a user $r$'s throughput is also
a concave function of $\mathbf{n}_r$.
%its strategy vector in this more general
%network.
This is summarized in Lemma \ref{lem:lem2}.

\begin{figure}[htb!]
    \begin{center}
    %%\includegraphics[width=0.5\textwidth, height=0.25\textheight]{./network1_2.eps}
    \includegraphics[scale=0.5]{eps_BT_research/network2.eps}
       \caption{A network where a user has multiple paths(or peers) to get data.}
\label{fig:network2}
    \end{center}
\end{figure}

\begin{lemma}
Suppose that user $r$ has $M$ ($M\ge 2$) paths in the network
shown in Figure \ref{fig:network2}, then the throughput of user
$r$ is a concave function of its strategy vector
$\mathbf{n}_r=(n_{r1}, n_{r2}, ..., n_{rM})$. \label{lem:lem2}
\end{lemma}



\begin{comment}
In the star topology, each user r has multiple paths to transfer
data to itself. Each path has two links. One link is the private
download link of user r, and the other link is the public upload
link of another peer. The private download link is shared by all
paths of user r. For example, in Figure \ref{fig:network2}, user
r's data receiving node is node A, its private link is link $B\to
A$. All paths, e.g., $C\to B\to A$ or $D\to B \to A$, share a
common link $B\to A$. Other than this common link, no any two
paths share any other common link.
\end{comment}

Based on Lemma \ref{lem:lem2}, we can show in the following
theorem the existence of NE on a star network.
%See \cite{zhang06tcpgame_general} for more details of the proof.
One
example of such star network is shown in Figure
\ref{fig:network3}.
%In this network, each user has two
%links connecting to the Internet. Out of these two links, one link
%is a upload link and the other link is download link. We assume
%that the download link of a user is a private link, that is, this
%download link can be used only by this user itself. On the other
%hand, the upload link of user $r$ are shared among all other users
%who are transferring data from user $r$.

\begin{theorem}
There exists a Nash equilibrium of unstructured overlay game on a
star network (shown in Figure \ref{fig:network3}).
\label{them:star}
\end{theorem}

%\begin{comment}

\begin{proof}
According to Lemma \ref{lem:lem2}, each user $r$'s total
throughput is a concave function of the vector of number of
connections $\mathbf{n_r}$. Then it is easy to show that user
$r$'s benefit or payoff function $B_r$ is a concave function of
$\mathbf{n_r}$. In addition, $B_r$ is continuous in $\mathbf{n}$.
Thus we have a multi-player \textit{concave} game. Based on the
result in \cite{rosen65}, we conclude that Nash equilibrium exists
in this game.
\end{proof}

%\end{comment}

\bigskip
\noindent \textbf{An illustrative example.} We use a simple star
network shown in Figure \ref{fig:star.simple.br} to illustrate the
existence of NE proved in Theorem \ref{them:star}. On this star
network, there are $6$ links $AD, DA, BD, DB, CD$, and $DC$. The
capacities of all links are $C_{AD}=10, C_{DA}=20, C_{BD}=30,
C_{DB}=40, C_{CD}=50$, and $C_{DC}=60$. There are three users
associated with nodes $A,B$ and $C$ respectively. For convenience,
we refer to the user at node $A$ as user $A$. Note that each user
has two download paths with each path consisting of two links. For
example, user $A$ has two download paths $B\to D \to A$ and $C\to
D \to A$. For any given download path, one link is shared with
other users, and the other link is a private link. For example,
for user $A$, path $B\to D \to A$ has two links: $B D$ and $DA$.
Link $BD$ is a link shared with user $C$.
%who has $n_{BC}$ number of connections to get data from
%$B$.
Link $DA$ is a private link of user $A$, which is shared by both
of its paths $B\to D \to A$ and $C\to D \to A$.

User $A$'s strategy is a vector of number of connections on two
available paths, i.e., $\mathbf{n}_A=(n_{BA}, n_{CA})$. Similarly,
strategies of user $B$ and $C$ are: $\mathbf{n}_B=(n_{AB},
n_{CB})$ and $\mathbf{n}_C=(n_{AC}, n_{BC})$.



Consider the unstructured file sharing game played by users $A,
B$, and $C$. Each user tries to maximize its benefit $B_r$
($r=A,B,C$). We use best response dynamics to demonstrate the
existence of a NE in this game. At the first step, each user opens
a random number of connections on two available paths. In the
following steps, only one player is randomly chosen to compute its
best response at each step. As shown in Figure
\ref{fig:star.simple.br}, the best response dynamics converges to
a NE, which can be verified by checking the optimality of benefits
of all three users.

\begin{figure}[htb!]
\centerline{
    \begin{minipage}{1.6in}
     \begin{center}
        \setlength{\epsfxsize}{1.6in}
        \epsffile{eps_BT_research/network6.eps}
    \end{center}
    \end{minipage}
    \begin{minipage}{1.6in}
      \begin{center}
        \setlength{\epsfxsize}{1.6in}
        \epsffile{eps_BT_research/star_simple_br_1.eps}
      \end{center}
    \end{minipage}
    }
\caption{The left figure shows a
       simple star topology with three users A, B, and C.
The right figure shows the best response dynamics. All three
users' benefits converge to the
        Nash equilibrium.}
        \label{fig:star.simple.br}
\end{figure}


%\noindent \textbf{

\subsection{Loss of Efficiency}

Consider the case where all downstream links have higher capacity
than upstream links and users are homogeneous. We can show that in
this case, the loss of efficiency of any NE in the game is
bounded. However, if users are aggressive in the sense that their
benefit functions do not contain cost terms, then a unique NE is a
point where all users open their maximum allowable number of
connections. Clearly, the loss of efficiency of the NE is
unbounded if users can open arbitrarily large numbers of
connections. In order to show these results, we need to do a
simple transformation as described below.

%For more information, please refer to our technical report
%\cite{zhang06tcpgame_general}.

In the star topology shown in Figure \ref{fig:network3}, if all
users' private downstream links have much higher capacities than
the upstream links of those other peers, then this game can be
thought of a variant of the game on a parallel link network. For
example, we can transform the simple star network in the left
sub-figure of Figure \ref{fig:star.simple.br} into Figure
\ref{fig:network8_2}. Center node $D$ in Figure
\ref{fig:star.simple.br} is decomposed into six interconnected
virtual nodes $D_{Ad},D_{Au},D_{Bd},D_{Bu},D_{Cd},D_{Cu}$. Links
between these six virtual nodes have infinite capacity. Node $A$
is decomposed into nodes $A_{down}$ and $A_{up}$. Link
$D_{Ad}A_{down}$ represents the downstream link of node $A$. Link
$A_{up}D_{Au}$ represents the upstream link of node $A$. Other
links have similar interpretations.


\begin{figure}[htb!]
    \begin{center}
    %%\includegraphics[width=0.5\textwidth, height=0.25\textheight]{./network1_2.eps}
    \includegraphics[scale=0.3]{eps_BT_research/network8_2.eps}
       \caption{Transformation of star network into equivalent parallel link network.}
\label{fig:network8_2}
    \end{center}
\end{figure}


Based on the transformation illustrated in Figure
\ref{fig:network8_2}, the result for the loss of efficiency at NE
on a parallel link network can be applied to a star network. That
is, the loss of efficiency at NE of the unstructured file sharing
game can be arbitrarily large if the number of users becomes large
in this special case.
%Thus, the price of anarchy (the worst case
%efficiency loss) is infinite for unstructured file sharing game.


%\begin{comment}

%\subsection{Price of Anarchy}


%\noindent \textbf{Benefit Including Throughput Only.}
We also consider another special case where users are aggressive
in the sense that users do not have cost constraint and only care
about their throughputs \cite{zhang05tcpgame_icnp}. That is,
user's benefit function is represented as: $
B_r(\mathbf{n_r})=G_r(\mathbf{n_r}). $ In this special case, there
exist a unique Nash equilibrium where all users open their maximum
allowable number of connections, and the price of anarchy can be
unbounded when users can open arbitrary large number of
connections.

%More details about the loss of efficiency on star network can be
%found in \cite{zhang06tcpgame_general}.

%\end{comment}
\bigskip
\noindent \textbf{Network Resource Utilization.} Suppose that all
downstream links have higher capacities than upstream links. Then
the capacities of all upstream links will be fully utilized at the
NE. This is a good situation in terms of the network resource
utilization because the total throughput can be supported by the
network is just the aggregate capacity of these upstream links.
Note that this is not always true for general network topologies,
%For counter examples, please see \cite{zhang06tcpgame_general}.
which is demonstrated in an example in Appendix
\ref{appd:resource.util}. A similar example is given in
\cite{key05fluid}.




%\subsection{Efficiency Loss of Nash Equilibrium}
%(...to be filled...)

%\section{Efficiency loss of un-coordinated congestion
%controllers}\label{sec:un-coord} (...to be filled...)



\section{Overlay Formation Game
% for BitTorrent-like P2P applications
} \label{sec:topo}

%\bigskip
%(... Giovanni's writeup...)
%\bigskip

%\input{giovanni_game}
\input{giovanni_network_game_new_2}
%\input{giovanni_network_game_new}

%\input{tcpgame-general-conclusion}

%\parskip=-10pt

\section{Conclusions}
\label{sec:con}

%\parskip=0pt

Motivated by unstructured file sharing networks such as BitTorrent
\cite{bt}, we introduced an unstructured file sharing game and an
overlay formation game to model the interaction among
self-interested users who can open multiple connections on
multiple paths to accelerate data transfer. Users are modelled as
players, and each user adjusts its numbers of connections on its
available paths to maximize its benefit.
%A user's benefit is a
%combination of utility (concave and nondecreasing function of a
%user's total throughput) and cost (proportional to the total
%number of connections of the user).
%The
%total throughput received by a user is the solution of the network
%optimization problem proposed by Kelly \cite{kelly98rate} and Mo
%and Walrand \cite{mo00fair}.
%So a user's benefit is a function of
%its numbers of connections on all paths.

%We first found that in order to find the maximum benefit
%%given
%%that all other users' numbers of connections are fixed,
%a user has to solve a Bi-level Programming problem, which is
%NP-hard on general networks. Then
We demonstrated by examples that there exist multiple stable
network states, so called Nash equilibria (NE), in the
unstructured file sharing game on general networks.
%, in order
%to make our models tractable,
We further restrict our attention to parallel link networks and
star networks which are used to model unstructured file sharing
networks. We proved the existence of NE in several variants of the
game on both networks. We found that the loss of efficiency of NE
can be arbitrarily large if users have no cost constraints.
However, when there are cost constraints, the loss of efficiency
is bounded. In addition, we proved the global stability of NE in
some variants of the game.
%We first investigated the unstructured file sharing game on
%parallel link networks, and proved the existence of Nash
%equilibrium. In order to quantify how bad a Nash equilibrium can
%be, we investigated the loss of efficiency of a Nash equilibrium,
%which is defined as the ratio of the maximum total benefit from
%all users over the total benefit at the Nash equilibrium.
Furthermore, we studied the Tit-for-Tat strategy (built in
BitTorrent \cite{bt})
%into the
%unstructured file sharing game to construct
through an overlay formation game.
%, in which a connection or
%virtual link between two users can be setup only when both users
%find this link beneficial.
%We used this game to model the built-in
%Tit-for-Tat mechanism in BitTorrent \cite{bt} applications.
We proved the existence of equilibrium overlays, and demonstrated
the convergence of the dynamical game-playing process. Although
the general belief is that the Tit-for-Tat can prevent selfish
behavior, we showed that it can still lead to an unbounded loss of
efficiency.



%\begin{comment}
%\parskip=-5pt
\section*{Acknowledgment}
%\parskip=0pt

The authors would like to thank Jim Kurose, Arun Venkataramani,
and Chun Zhang for their helps. This research has been supported
in part by NSF under grant awards ANI-0085848, CNS-0519998,
CNS-0519922, and EIA-0080119, and by Italian MIUR project Famous.
Any opinions, findings, and conclusions or recommendations
expressed in this material are those of the authors and do not
necessarily reflect the views of the National Science Foundation.

%\end{comment}

%\parskip=-5pt

%\bibliographystyle{abbrv}
\bibliographystyle{IEEEtran}
\bibliography{tcpgame-general-tech}

%\section{Appendix}
\appendices

\bigskip
\section{An example for the simple rate allocation mechanism}\label{appd:simple}

This example is to show that the simple rate allocation mechanism
in Section \ref{sec:rate.alloc} cannot be extended to a general
network.

Suppose that there are two paths $p_1$ and $p_2$ which belong to
two user $r1$ and $r2$ respectively. These two paths share a
single common link $l$. Let user $r1$ open $n_{1p_1}$ number of
connections on $p_1$ and user 2 open $n_{2p_2}$ number of
connections on $p_2$.

Suppose that
$$\forall m\in p_1 , m\ne l,
C_m n_{1p_1}/\sum_{w\in \mathbf{R}} n_{wm} > C_l
n_{1p_1}/(n_{1p_1}+n_{2p_2}).$$
%Here, we have $n^1_l=n^1_m=n^1$.
Here, $n_{wm}$ represents the number of connections opened by user
$w$ on link $m$.

If we conclude that
$$y_{1p_1}=C_l n_{1p_1}/(n_{1p_1}+n_{2p_2}),$$ then we might be wrong.
The reason is as follows.

It is possible that there is
\begin{comment}
$$
\forall m\in p_2, m\ne l,
 C_k n^2/\sum_{w\in \mathbf{R}} n^w_k < C_l
n^2/\sum_{w\in \mathbf{R}} n^w_l.
$$
\end{comment}
a link $k$ on $p_2$ satisfying
$$
 k\in p_2, k\ne l,
 C_k n_{2p_2}/\sum_{w\in \mathbf{R}} n_{wk} \le C_j
n_{2p_2}/\sum_{w\in \mathbf{R}} n_{wj}, \forall j\in p_2,
$$
then user $r2$'s obtained rate is
$$
C_k n_{2p_2}/\sum_{w\in \mathbf{R}} n_{wk},
$$
and the actual allocated rate of user $r2$ on link $l$ is $C_k
n_{2p_2}/\sum_{w\in \mathbf{R}} n_{wk}$.

If we have
\begin{eqnarray}
C_m n_{1p_1}/\sum_{w\in \mathbf{R}} n_{wm} &>& C_l - C_k
n_{2p_2}/\sum_{w\in \mathbf{R}}n_{wk} \nonumber \\
&>& C_l n_{1p_1}/(n_{1p_1}+n_{2p_2}) \nonumber
\end{eqnarray}
then, the actual rate obtained by user 1 is
$$
C_l - C_k n_{2p_2}/\sum_{w\in \mathbf{R}} n_{wk},
$$
not $$C_l n_{1p_1}/(n_{1p_1}+n_{2p_2}).$$

\bigskip
\section{Proof of Theorem \ref{them:con}}\label{appd:con}


Consider the Lagrangian of the constrained optimization problem of
any given user $r$.
$$
L(\mathbf{n}_r)=B_r(\mathbf{n}_r) + \lambda (n_r - \sum_{j=1}^L
n_{rj}).
$$

The optimal solution can be obtained by solving the following
equations.
\begin{eqnarray}
\partial{L}/\partial{n_{rj}} &=&0, \forall j \\
\partial{L}/\partial{\lambda} &=& 0
\end{eqnarray}
That is,
\begin{eqnarray}
\partial{L}/\partial{n_{rj}} &=& \frac{\sum_{k\ne r}^R n_{kj}}{
(n_{rj}+\sum_{k\ne r}^R n_{kj})^2} C_j - \beta - \lambda=0 \\
\partial{L}/\partial{\lambda} &=& n_r -
\sum_{j=1}^L n_{rj}=0
\end{eqnarray}

We consider a symmetric Nash equilibrium where all users have the
same number of connections on each path/link. Then, we get
$$
C_i/n^*_{ri} = C_j/n^*_{rj}, \forall i, j
$$
Combined with $\sum_{j=1}^L n^*_{rj}=n_r$, we can get the vector
of number of flows at Nash equilibrium. Specifically, for a given
user $r$, its number of connections at link $j$ at Nash
equilibrium is given by $n^*_{rj}=n_r C_j/\sum_{k=1}^L C_k.$

\bigskip
\section{An example for under-utilized network resources
due to selfish behavior of users.  } \label{appd:resource.util}

In the triangle network shown in Figure \ref{fig:triangle_bw},
consider that all links are bi-directional and all links have the
same capacity $C$. We have
$C_{AB}=C_{BA}=C_{AC}=C_{CA}=C_{BC}=C_{CB}=C$. There are six
users:
\begin{itemize}
\item User $AB$ wants to transfer data from node $A$ to node $B$.
\item User $BA$ wants to transfer data from node $B$ to node $A$.
\item User $BC$ wants to transfer data from node $B$ to node $C$.
\item User $CB$ wants to transfer data from node $C$ to node $B$.
\item User $AC$ wants to transfer data from node $A$ to node $C$.
\item User $CA$ wants to transfer data from node $C$ to node $A$.
\end{itemize}

Consider that each user has two paths to transfer data and it can
only open at most one connection on each path. For clarity, in
Figure \ref{fig:triangle_bw}, we only show connections opened by
user $AB$ and user $BA$. Assume that all users try to maximize its
total throughput, then at the NE, every user opens one connection
on each of its two paths. Each user gets a total throughput of
$2C/3(=C/3+C/3)$. Then, the total throughput from all six users is
$4C$. However, the total capacity provided by the network is $6C$.
Thus, the network resource is not fully utilized in this example.
A similar example is given in \cite{key05fluid}.


\begin{figure}[htb!]
    \begin{center}
    %%\includegraphics[width=0.5\textwidth, height=0.25\textheight]{./network1_2.eps}
    \includegraphics[scale=0.6]{eps_BT_research/triangle_bw.eps}
       \caption{Triangle Network.}
\label{fig:triangle_bw}
    \end{center}
\end{figure}



\input{appendix-giovanni}

% trigger a \newpage just before the given reference
% number - used to balance the columns on the last page
% adjust value as needed - may need to be readjusted if
% the document is modified later
%\IEEEtriggeratref{8}
% The "triggered" command can be changed if desired:
%\IEEEtriggercmd{\enlargethispage{-5in}}

% references section
% NOTE: BibTeX documentation can be easily obtained at:
% http://www.ctan.org/tex-archive/biblio/bibtex/contrib/doc/

% can use a bibliography generated by BibTeX as a .bbl file
% standard IEEE bibliography style from:
% http://www.ctan.org/tex-archive/macros/latex/contrib/supported/IEEEtran/bibtex
%\bibliographystyle{IEEEtran.bst}
% argument is your BibTeX string definitions and bibliography database(s)
%\bibliography{IEEEabrv,../bib/paper}
%
% <OR> manually copy in the resultant .bbl file
% set second argument of \begin to the number of references
% (used to reserve space for the reference number labels box)


% that's all folks
\end{document}
