\documentclass{seminar}
\usepackage{doublespace}
\usepackage{times}
\usepackage{smallheadings}
\usepackage{subfigure}
\usepackage{amsfonts}
\usepackage{graphicx}
\usepackage{multicol}
\usepackage{listings}
\input{epsf}

\usepackage{graphics}

\begin{document}



\begin{slide}
{\bf Solving  homogeneous recurrences} \\
Given $a_0t_n+a_1t_{n-1}+...+a_kt_{n-k}=0$. Guess $t_n=x^n$ for an unknown
constant $x$.\\
We have  $a_0x^n+a_1x^{n-1}+...+a_kx^{n-k}=0$ \\
Ignoring solution $x=0$, the equation is satisfied if and only if \\
$p(x)=a_0x^k+a_1x^{k-1}+...+a_k=0$
It is called the \em{characteristic equation} of the recurrence. \\

Let $r_1, r_2,..., r_k$ be the $k$ roots of $p(x)$. We conclude that
$t_n=\sum_{i=1}^{k}{c_ir_i^n}$ for any constants $c_i$.

This is the only solution when the $k$ roots are distinct.
\end{slide}

\begin{slide}
{\bf Example} \\
\[f_n = \left \{
         \begin{array}{ll}
         n, & n=0,1\\
         f_{n-1} + f_{n-2}, & otherwise\\
	 \end{array}
        \right.
\]

We have $f_n - f_{n-1} - f_{n-2} =0$.
The characteristic equation is $x^2 -  x -1 =0$ \\

whose roots are
$r_1=\frac{1+\sqrt{5}}{2}$ and $r_2=\frac{1-\sqrt{5}}{2}$.

So $f_n = c_1*(\frac{1+\sqrt{5}}{2})^n + c_2(\frac{1-\sqrt{5}}{2})^n$.

We know $f_0=0=c_1+c_2$ and
$f_1=1=c_1*\frac{1+\sqrt{5}}{2}+c_2\frac{1-\sqrt{5}}{2}$

We have $c_1=1/\sqrt{5}$ and $c_2=-1/\sqrt{5}$.
\end{slide}

\begin{slide}
{\bf Multiple roots} \\
In general, if $r_1, r_2, ...,r_l$ are the $l$ distinct roots of the
characteristic polynomial and their multiplicities are $m_1,m_2,...,m_l$, 
then $$t_n = \sum_{i=1}^{l}\sum_{j=0}^{m_i-1}c_{ij} n^j r_i^n$$.
\end{slide}

\begin{slide}
{\bf Example: multiple roots}
\[ 
t_n = \left\{ 
  \begin{array}{ll}
     n, & n=0,1,or \; 2 \\
     5t_{n-1}-8t_{n-2}+4t_{n-3} & \\
   \end{array}
\right.
\]

The characteristic polynomial is $$x^3-5x^2+8x-4=(x-1)(x-2)^2$$.

So $t_n=c_1 1^n + c_2 2^n + c_3 n 2^n$.
Applying the initial conditions, we obtain $c_1=-2, c_2=2$ and $c_3=-1/2$. 
Therefore $t_n=2^{n+1}-n 2^{n-1}-2$.
\end{slide}

\begin{slide}
{\bf Inhomogeneous recurrences: a general form}\\
Consider the following generalization
$$a_0t_n+a_1t_{n-1}+...+a_kt_{n-k}=b_1^n p_1(n)+b_2^n p_2(n)+...,$$
where $b_i$ is a constant and $p_i(n)$ is a polynomial in $n$ of degree $d_i$.

The characteristic polynomial is 
$$(a_0x^k+a_1x^{k-1}+...+a_k)\Pi_{i}(x-b_i)^{d_i+1}.$$
\end{slide}

\begin{slide}
{\bf Inhomogeneous recurrences: an example}\\
\[
t_n = \left\{
 \begin{array}{ll}
       0, & n=0 \\
       2t_{n-1}+n+2^n & otherwise \\
 \end{array}
 \right.
\]
Rewrite the recurrence as $$t_n-2t_{n-1}=1^nn^1+2^nn^0$$
So $b_1=1, p_1(n)=n, b_2=2,$ and $p_2(n)=1$.
The characteristic polynomial is
$$(x-2)(x-1)^2(x-2)=(x-1)^2(x-2)^2,$$
All solutions to the recurrence has the form\\
$$t_n=c_11^n+c_2n1^n+c_32^n+c_4n2^n$$
Substitute it into the original recurrence, which gives
$$n+2^n=(2c_2-c_1)-c_2n+c_42^n$$.
We obtain $c_4=1$, and thus $t_n=\Theta(n2^n)$.
\end{slide}

\begin{slide}
{\bf Master Theorem: a simple version} \\
Let $T: N \rightarrow R^+$ be an eventually nondecreasing function such that
$$ T(n) = lT(n/b)+cn^k, n>n_0$$ when $n/n_0$ is an exact power of $b$.
The constants $n_0, l \ge 1, b \ge 2$, and $k \ge 0$ are all integers. $c$ is a
positive real number.

We have 
\[
T(n) \in \left\{
  \begin{array}{lll}
  \Theta(n^k) &if&  l<b^k \\
  \Theta(n^k logn) &if&  l=b^k \\
  \Theta(n^{log_{b}{l}}) &if&  l>b^k \\
  \end{array}
  \right.
\]
\end{slide}

\begin{slide}
{\bf Examples}
$$T(n) = 9T(n/3)+n.$$
\end{slide} 

\end{document}
