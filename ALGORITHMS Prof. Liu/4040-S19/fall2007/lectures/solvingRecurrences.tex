\documentclass{seminar}
%\usepackage{doublespace}
\usepackage{times}
%\usepackage{smallheadings}
\usepackage{subfigure}
\usepackage{amsfonts}
\usepackage{graphicx}
\usepackage{multicol}
\usepackage{listings}
\input{epsf}

\usepackage{graphics}

\begin{document}

\begin{slide}
{\bf Recurrences}
\begin{itemize}
\item The substitution method
\item The recursion tree method
\item The master method
\end{itemize}
\end{slide}
\begin{slide}
{\bf The substitution method}
\begin{itemize}
\item Guessing the form of the solution
\item Using the mathematical induction to show that the solution works
\end{itemize}
\end{slide}

\begin{slide}
{\bf The substitution method: an example} \\
We'd like to solve $T(n)=3T(\lfloor n/4 \rfloor)+n$. \\
We guess $T(n) \in O(n)$. \\
We prove by induction that there exists a constant $c$ such that $T(n)
\le cn $ for sufficiently large $n$.

\begin{eqnarray*}
T(n) & = & n + 3T(\lfloor n/4 \rfloor) \\ 
     & \le & n + 3*c*\lfloor n/4 \rfloor \\
     & \le & (1+3c/4)n  \\
     & \le & cn, \; when \; c \ge 4 \\
\end{eqnarray*}
\end{slide}


\begin{slide}
{\bf The substitution method: subtleties} \\
We'd like to solve $T(n)=T(\lfloor n/2 \rfloor)+ T(\lceil n/2 \rceil) + 1$. \\
We guess $T(n) \in O(n)$. \\
We try to prove by induction that there exists a constant $c$ such that $T(n)
\le cn $ for sufficiently large $n$.

\begin{eqnarray*}
T(n) & = & T(\lfloor n/2 \rfloor)+ T(\lceil n/2 \rceil) + 1 \\ 
     & \le & c*\lfloor n/2 \rfloor + c*\lceil n/2 \rceil + 1\\
     & = & cn + 1  \\
\end{eqnarray*}
We really need to show that $T(n) \le cn$.
\end{slide}

\begin{slide}
{\bf The trick} \\
We'd like to solve $T(n)=T(\lfloor n/2 \rfloor)+ T(\lceil n/2 \rceil) + 1$. \\
We guess $T(n) \in O(n)$. \\
We  prove by induction that there exists a constant $c,b$ such that $T(n)
\le cn-b$ for sufficiently large $n$.

\begin{eqnarray*}
T(n) & = & T(\lfloor n/2 \rfloor)+ T(\lceil n/2 \rceil) + 1 \\
     & \le & c*\lfloor n/2 \rfloor-b + c*\lceil n/2 \rceil -b + 1\\
     & = & cn - b - (b-1)  \\
     & < & cn-b \; when \; b>1
\end{eqnarray*}
\end{slide}


\begin{slide}
{\bf The  recursion-tree method}
\begin{itemize}
\item The method
  \begin{itemize}
  \item Draw a recursion tree where each node represents the cost of a single
    subproblem
  \item Sum the cost of each level to get per-level cost
  \item Sum all per-level costs to get the total cost
  \end{itemize}

\item Applications
\begin{itemize}
\item Can be used to find a good guess. Complete by using the substitution
  method. Can be a bit sloppy when constructing the tree. 
\item Can serve as a direct proof. Need to be strict when draw the tree.
\end{itemize}
\end{itemize}
\end{slide}


\begin{slide}
{\bf The recursion-tree method: an example} \\
We'd like to solve $T(n)=3T(\lfloor n/4 \rfloor)+ \Theta(n^2)$. \\
We instead draw a tree for $T(n)=3T(n/4)+ cn^2$. \\
Some sloppiness we use here
\begin{itemize}
\item assume $n$ is an exact power of 4 to remove the floor function
\item replace $\Theta(n^2)$ by $cn^2$ 
\end{itemize}
\end{slide}

\begin{slide}
The sum of per-level costs results below:

\begin{eqnarray*}
T(n) &=& cn^2 + {3 \over 16}cn^2 + ({3 \over 16})^2cn^2+...+  ({3 \over
  16})^{log_{4}n-1}cn^2 + 3^{log_{4}{n}}\Theta(1) \\
     &=& \sum_{i=0}^{log_{4}n-1}{(\frac{3}{16})^i}cn^2 +
\Theta(n^{log_{4}{3}}) \\
      &<& \sum_{i=0}^{\infty}{(\frac{3}{16})^i}cn^2 +
\Theta(n^{log_{4}{3}}) \\
     &=& \frac{1}{1-3/16}cn^2+\Theta(n^{log_{4}{3}})\\
     &=& O(n^2) 
\end{eqnarray*}
\end{slide}


\begin{slide}
{\bf Master Theorem: a simple version} \\
Let $T: N \rightarrow R^+$ be an eventually non-decreasing function such that
$$ T(n) = lT(n/b)+cn^k, n>n_0$$ when $n/n_0$ is an exact power of $b$.
The constants $n_0, l \ge 1, b \ge 2$, and $k \ge 0$ are all integers. $c$ is a
positive real number.

We have 
\[
T(n) \in \left\{
  \begin{array}{lll}
  \Theta(n^k) &if&  k > log_bl \\
  \Theta(n^k logn) &if& k = log_bl \\
  \Theta(n^{log_{b}{l}}) &if&  k < log_bl \\
  \end{array}
  \right.
\]
\end{slide}

\begin{slide}
{\bf Asymptotic recurrences} \\
Consider a function $T: N \rightarrow R^+$ such that
$$T(n) = lT(n/b) + f(n)$$
for all sufficiently large n, where $l \ge 1$ and $b \ge 2$ are constants, and 
$f(n) \in \Theta(n^k)$ for some $k \ge 0$. We conclude that 

\[
T(n) \in \left\{
 \begin{array}{lll}
       \Theta(n^k)& if & k > log_bl \\
       \Theta(n^k\log{n})& if & k = log_bl \\
       \Theta(n^{\log_{b}{l}}) & if & k < log_bl \\
 \end{array}
 \right.
\] 
 
\end{slide}

\begin{slide}
{\bf Master Theorem} \\
Let $l \ge 1$ and $b>1$ be constants, let $f(n)$ be a function, and $T(n)$ be
defined on the non-negative integers $n$ by the recurrence
$$T(n) = lT(n/b)+f(n),$$
where we interpret $n/b$ be either $\lfloor n/b \rfloor$ or $\lceil n/b
\rceil$. The $T(n)$ can be bound asymptotically as follows.

\begin{enumerate}
\item If $f(n) = O(n^{log_{b}{l} - \epsilon})$ for some constant $\epsilon > 0$,
  then $T(n) = \Theta(n^{log_{b}{l}})$.

\item If $f(n) = \Theta(n^{log_{b}{l}})$,
  then $T(n) = \Theta(n^{log_{b}{l}} lg n)$.

\item If $f(n) = \Omega(n^{log_{b}{l}  + \epsilon})$ for some constant $\epsilon > 0$,
and if $lf(n/b) \le cf(n)$ for some constant $c < 1$ and all sufficiently
large $n$, then $T(n) = \Theta(f(n))$.
\end{enumerate}
\end{slide}

\begin{slide}
{\bf Examples}
$$T(n) = T(n/3)+1.$$

$$T(n) = T(n/3)+n. $$

$$T(n) = 9T(n/3)+n.$$

$$T(n) = 3T(n/4) + nlg n.$$
\end{slide} 

\begin{slide}
{\bf Change variables}\\
Consider the recurrence $T(n)=2T(\sqrt{n})+ lgn$. \\
Let $ n =2^m$, then $T(2^m) = 2T(2^{m/2})+m$ \\
Let $S(m)$ denote $T(2^m)$, then $S(m) = 2 S(m/2) + m$\\

Using the master method, $S(m) \in \Theta(m lg m)$
So $T(n) \in \Theta(lg n lg lg n) $.
\end{slide}

\begin{slide}
{\bf Range transformations} \\
Consider the following recurrence where n is a power of 2.
\[
T(n) = \left\{
 \begin{array}{ll}
       1/3, & n=1 \\
       nT^2(n/2) & otherwise \\
 \end{array}
 \right.
\] 

Let $t_i$ denote $T(2^i)$.
$$t_i=T(2^i)=2^iT^2(2^{i-1})=2^it^2_{i-1}$$

Let $u_i$ denote $lg t_i$.
$$u_i = lg t_i = lg(2^it^2_{i-1}) = i + 2 lgt_{i-1} = i+ 2u_{i-1}$$
We solve this equation using recursion tree.
\end{slide}

\end{document}
