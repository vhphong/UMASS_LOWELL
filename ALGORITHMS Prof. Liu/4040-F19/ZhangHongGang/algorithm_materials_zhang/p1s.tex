\documentclass[11pt]{article}

\setlength {\oddsidemargin}{0.5in} \setlength
{\evensidemargin}{0.5in} \setlength {\textwidth}{5.5in}


\setlength{\parindent}{0.0in} \setlength{\parskip}{12pt}
\setlength{\topmargin}{-0.35in} \setlength{\textheight}{8.5in}
\setlength{\oddsidemargin}{-0.25in}
\setlength{\evensidemargin}{0pt} \setlength{\textwidth}{6.5in}
\def\singlespace{\baselineskip=1em}
\def\doublespace{\baselineskip=2em}

\def\blank#1{$\underline{\hbox to #1{\hfil}}$}

\begin{document}
%\begin{flushright}
%\today
%\end{flushright}

\begin{center}
{\bf CMPSC 623 Problem Set 1} \\
{\bf by Prof. Honggang Zhang} \\
\end{center}


\begin{description}

\item[Problem 1.]
 Exercise 1.2-2 on page 13. In addition, how might one
rewrite the merge sort pseudocode to make it even faster on small
inputs?

\noindent \textbf{Solution:}

Insertion sort beats merge sort when $8n^2<64n\lg n$, so we know
that $n< 8 \lg n$, and $2^{n/8} < n$, which is true when $2\le n
\le 43$.

Merge sort can be rewritten so that it does insertion sort on
inputs of size 43 or less. The modified merge sort now takes fewer
steps than the straight merge sort.

\item[Problem 2.] Rank the following functions by order of growth; that is,
find an arrangement $g_1, g_2, ..., g_{23}$ of the functions
satisfying $g_1=\Omega, g_2=\Omega(g_3), ...,
g_{22}=\Omega(g_{23})$. Partition your list into equivalent
classes such that $f(n)$ and $g(n)$ are in the same class if and
only if $f(n)=\Theta(g(n))$.
$$(3/2)^n, (\sqrt{2})^{\lg n}, \lg^*n, n^2, n^3, \lg^2n,
\lg (n!), 2^{2^n}, n^{1/\lg n}, \lg \lg n, n\cdot 2^n, n^{\lg \lg
n}
$$
$$
\ln n, 2^n, 2^{\lg n}, (\lg n)^{\lg n}, 4^{\lg n}, (n+1)!,
\sqrt{\lg n}, n!,  n, n\lg n, 1
$$

\noindent \textbf{Solution:}

$$
2^{2^n} \quad (n+1)! \quad n! \quad n\cdot 2^n \quad 2^n \quad
(3/2)^n \quad (\lg n)^{\lg n}=n^{\lg \lg n} \quad n^3 \quad
n^2=4^{\lg n} \quad
$$
$$
n\lg n =\Theta(\lg (n!)) \quad n=2^{\lg n} \quad (\sqrt{2})^{\lg
n} \quad \lg^2n \quad \ln n \quad \sqrt{\lg n} \quad \lg \lg n
\quad \lg^*n \quad n^{1/\lg n}= \Theta(1)
$$

Note, $(\lg n)^{\lg n}=n^{\lg \lg n}$. $n^{1/\lg n}=2$ because
$2^{\lg n}=n$.

\item[Problem 3.] Problem 2-1 on page 37.

\noindent \textbf{Solution:}

a)\\
Insertion sort takes $\Theta(k^2)$ time per k-element list in the
worst case. Therefore, sorting $n/k$ such k-element lists takes
$\Theta(k^2n/k)=\Theta(nk)$ worst-case time.

b)\\
%Just extending the 2-list merge to merge all the lists at once
%would take $\Theta(n(n/k))$ times (n from copying each element
%once into the result list, $n/k$ from examining $n/k$ lists at
%each step to select next item for result list).

The key idea is to merge the lists pairwise, then merge the
resulting lists pairwise, and so on, until there is just one list.
The pairwise merging requires $\Theta(n)$ work at each level,
since we are sill working on n elements, even if they are
partitioned among sublists. The number of levels, staring with
$n/k$ k-element lists and finishing with 1 n-element list, is
$\lceil \lg(n/k) \rceil$. Therefore, the total running time for
the merging is $\Theta(n\lg (n/k))$.

c)\\
The modified algorithm has the same asymptotic running time as
standard merge sort when
$$
\Theta(nk+n\lg(n/k))=\Theta(n\lg n).
$$
The largest asymptotic value of k as a function of n that
satisfies this is $k=\Theta(\lg n)$.

k can't be more than $\Theta(\lg n)$  or the left-hand expression
would not be $\Theta(n\lg n)$ (because it would have a
higher-order term than $n\lg n$). So all we need to do is verify
that $k=\Theta(\lg n)$ works, which we can do by plugging $ k=\lg
n$ into
$$
\Theta(nk+n\lg (n/k))=\Theta(nk+n\lg n - n\lg k)
$$
to get
$$
\Theta(n\lg n + n\lg n - n\lg \lg n)=\Theta(2n\lg n - n\lg \lg n)
$$
Then by taking just the high-order term and ignoring the constant
we can get $\Theta(n\lg n)$.

d)\\
In practice, k should be the largest list length on which
insertion sort is faster than merge sort.

\item[Problem 4.] Exercise 3.1-1 on page 50.

\noindent \textbf{Solution:}

Let $h(n)=\max\{f(n), g(n)\}$. Then
$$
h(n) = \left\{ \begin{array}{ll}
f(n) & \textrm{if $f(n)\ge g(n)$}\\
g(n) & \textrm{if $f(n)< g(n)$}
\end{array} \right.
$$
As $f(n)$ and $g(n)$ are asymptotically non-negative, there exists
$n_0$ such that $f(n)\ge 0$ and $g(n)\ge 0, \forall n\ge n_0$.
Thus, for $n\ge n_0, f(n)+g(n) \ge f(n) \ge 0,  f(n)+g(n) \ge g(n)
\ge 0$. Because $h(n)$ is either $f(n)$ or $g(n)$, we have
$$
f(n) + g(n) \ge h(n) \ge 0
$$
which shows that
$$
h(n)=\max (f(n), g(n))= O(f(n) +g(n)).
$$

Similarly, since $h(n)$ is the larger of $f(n)$ and $g(n)$, we
have $\forall n\ge n_0, 0\le f(n)\le h(n)$ and $0 \le g(n) \le
h(n)$. Thus, we have
$$
0\le f(n) + g(n) \le 2h(n)
$$
or
$$
0\le 1/2 \cdot(f(n) + g(n)) \le h(n)
$$
This means that $h(n)=\Omega(f(n)+g(n))$, with constant $c=1/2$ in
the definition of $\Omega$.

$h(n)=\Omega(f(n)+g(n))$ and $h(n)=O(f(n)+g(n))$ indicates that
$h(n)=\Theta(f(n)+g(n))$.

Proved.


%\item[Problem 4.] Exercise 3.1-2  on page 50.
\item[Problem 5.] Exercise 3.1-3  on page 50.

\noindent \textbf{Solution:}

Let the running time be $T(n)$. $T(n)\ge O(n^2)$ means that
$T(n)\ge f(n)$ for some function $f(n)$ in the set $O(n^2)$. This
is true for any $T(n)$ since the function $g(n)=0, \forall n$ is
in $O(n^2)$ and $T(n)$ is always larger than 0. So this statement
tells us nothing about the running time.

%\item[Problem 6.]Exercise 3.1-4  on page 50.
\item[Problem 6.]Exercise 3.1-8  on page 50.

\noindent \textbf{Solution:}

\begin{eqnarray}
\Omega(g(n,m))=\{ f(n,m) &: \mbox{there exist positive
constants}\quad c, n_0, \quad \mbox{and} \quad m_0 \nonumber \\
&\mbox{such that}\quad 0\le cg(n,m) \le f(n,m) \quad \nonumber \\
&\mbox{for all} \quad n\ge n_0 \quad \mbox{and}\quad m\ge
m_0.\}\nonumber
\end{eqnarray}

\begin{eqnarray}
\Theta(g(n,m))=\{ f(n,m) &: \mbox{there exist positive
constants}\quad c_1, c_2, n_0, \quad \mbox{and} \quad m_0 \nonumber \\
&\mbox{such that}\quad 0\le c_1 g(n,m)\le f(n,m) \le c_2 g(n,m) \nonumber \\
&\mbox{ for all} \quad n\ge n_0 \quad\mbox{ and }\quad m\ge m_0.\}
\nonumber
\end{eqnarray}

%\item[Problem 8.] Exercise 3.2-4 on page 57.


\end{description}

\end{document}
