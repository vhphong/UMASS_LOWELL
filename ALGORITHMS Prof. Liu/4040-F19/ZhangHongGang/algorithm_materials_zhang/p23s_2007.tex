\documentclass[11pt]{article}

\setlength {\oddsidemargin}{0.5in} \setlength
{\evensidemargin}{0.5in} \setlength {\textwidth}{5.5in}


\setlength{\parindent}{0.0in} \setlength{\parskip}{12pt}
\setlength{\topmargin}{-0.35in} \setlength{\textheight}{8.5in}
\setlength{\oddsidemargin}{-0.25in}
\setlength{\evensidemargin}{0pt} \setlength{\textwidth}{6.5in}
\def\singlespace{\baselineskip=1em}
\def\doublespace{\baselineskip=2em}

\def\blank#1{$\underline{\hbox to #1{\hfil}}$}

\begin{document}
%\begin{flushright}
%\today
%\end{flushright}

\begin{center}
{\bf CMPSC 623 Problem Set 2 and 3} \\
{\bf Solutions to problems not discussed in class.}\\
{\bf by Prof. Honggang Zhang} \\
\end{center}


\begin{description}

%\item[Problem 1.]
%Solve (find an asymptotic upper bound for) the recurrence
%$$
%T(n)=T(n-a)+T(a)+n
%$$
%where $a\ge 1$ is constant, by using iteration to generate a guess
%and then using substitution method to verify it.


\item[HW-2 Problem 8.]
(Probabilistic Analysis) Page 118, Problem 5-2. Only do part (a), (b), (c).

Part (a). Simply transfer the description into code. Need to keep
an array for recording if a particular item is checked. Also need
a counter to record the number of distinct elements that have been
checked.

Part (b). Suppose we want to search for $x$ whose index is $i$,
that is, $A[i]=x$. We are doing a random search, so at each probe,
any element is equally likely to be picked up. Then, the
probability that we find the target $x$ at the first pick or probe
is simply $1/n$. The prob. that we find it at the second pick is
$(n-1)/n * 1/n$ (this means, we are not successful at the first
probe but successful at the second probe). Let $X$ denote the
number of probes until we find the target. Then it is easy to see
that $X$ is a geometric random variable. Then $E[X]=1/(1/n)=n$.

Part (c). Element $x$ now has $k$ replicas in the total $n$
elements. Again, we assume that each elem. is equally likely to be
picked up at a probe. Then at the first probe, we can find x with
probability $k/n$. Similar to Part (b), the probability of being
successful at the second probe is $(n-k)/n * k/n$, so on so forth.
$X$ is now a geometric random variable with $p=k/n$. Then
$E[X]=n/k$.

\item[Hw-2 Problem 10.] Page 160, Problem 7-2. Only do part (a), (b), (c), and (e). Note, you need to
use substitution method for (e), and you can use the result in
part (d) without proof.

We have discussed Part a,b,c in class.

Part (e) is a little tricky. Different prints of the 2nd edition
of the textbook give you different problem descriptions and hints!
So I didn't grade it. One hint is a tricky way to handle
substitution method. I don't recall that I discussed it in class
when I was discussing substitution method. So I didn't grade it.
In any case, here is the idea:

For inductive step, you assume that $ET(n)\le an\log n - bn, a>0,
b>0, \forall n<k$. Then you want to prove that $ET(n)\le an\log n
- bn$.

\begin{eqnarray}
ET(n) &=& 2/n \sum_{q=2}^{n-1} ET(n) + \Theta(n) \qquad \mbox{(from part (c))}\\
&\le& 2/n \sum (aq\log q - bq) + \Theta(n)  \qquad\mbox{(from our inductive hypothesis)}\\
&\le& 2/n (1/2 an^2\log n - a/8 n^2 - b(n-1)n/2) + \Theta(n) \qquad\mbox{(from part (d))}\\
&\le& an\log n - bn + b + \Theta(n)\\
&\le& an\log n - bn - (a/4 n - cn - b)
\end{eqnarray}
In the above, we let $\Theta(n)=cn$. Then, one can find $a\ge 4 c
+ 4b/n$. One also needs to check base case.








\item[HW-3 Problem 5.] Problem 7-4 on page 162.

a). discussed in class.

b).

The stack depth of $QUICKSORT'$ will be $\Theta(n)$ on an n-element input array if there are
$\Theta(n)$ recursive calls to $QUICKSORT'$. This happens if every call to $PARTITION(A,p,r)$
returns $q=r$. The sequence of recursive calls in this scenario is:
$$
QUICKSORT' (A,1,n), QUICKSORT' (A,1,n-1), QUICKSORT' (A,1,n-2),
$$
$$
 ... QUICKSORT' (A,1,1).
$$

As we discussed in class, applying the partition procedure
discussed in class to a sorted list will always produce the worst
stack depth.

 c).

The problem demonstrated by the scenario in (b) is that each call
of $QUICKSORT'$ calls $QUICKSORT'$ again with almost the same
range (size only reduced by one). To avoid such behavior, we must
change $QUICKSORT'$ so that recursive call is on a smaller
interval of the array. The following variation of $QUICKSORT'$
checks which of the two sub-arrays returned from $PARTITION$ is
smaller and recurses on the smaller sub-array, which is at most
half the size of the current array. Since the array size is
reduced by at least half on each recursive call, the number of
recursive calls, and hence the stack depth, is $\Theta(\lg n)$ in
the worst case. Note that this works no matter what PARTITION
algorithms we use.

Note that a rough even split at each step produces the worst case
stack depth. Compare this with the regular quicksort which has the
best case running time when each split is a even split.


\begin{figure}[|htb]
\begin{center}
\begin{tabular}{|l|}\hline
$QUICKSORT''(A,p,r)$\\
\makebox[.6cm][l]{\hspace*{.2em}1. } while $p<r$\\
\makebox[.6cm][l]{\hspace*{.2em}2. } \hspace{.1in} do // Partition and sort the
smaller subarray first\\
\makebox[.6cm][l]{\hspace*{.2em}3. }\hspace{.5in}$q\gets$
$PARTITION(A,p,r)$\\
\makebox[.6cm][l]{\hspace*{.2em}4. }\hspace{.5in}if $q-p <
r-q$ \\
\makebox[.6cm][l]{\hspace*{.2em}5. }\hspace{.7in} then
$QUICKSORT''(A,p,q-1)$\\
\makebox[.6cm][l]{\hspace*{.2em}6. }\hspace{.9in}$p\gets q+1$\\
\makebox[.6cm][l]{\hspace*{.2em}7. }\hspace{.7in}else
$QUICKSORT''(A,q+1, r)$\\
\makebox[.6cm][l]{\hspace*{.2em}8. }\hspace{.9in} $r\gets q-1$\\
\hline
\end{tabular}
\end{center}
\end{figure}

The expected running time is not affected, because exactly the
same work is done as before: The same partitions are produced, and
the same sub-arrays are sorted.



\end{description}

\end{document}
