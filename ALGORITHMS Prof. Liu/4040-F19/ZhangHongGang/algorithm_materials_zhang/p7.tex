\documentclass[11pt]{article}


\setlength {\oddsidemargin}{0.5in} \setlength
{\evensidemargin}{0.5in} \setlength {\textwidth}{5.5in}


\setlength{\parindent}{0.0in} \setlength{\parskip}{12pt}
\setlength{\topmargin}{-0.35in} \setlength{\textheight}{8.5in}
\setlength{\oddsidemargin}{-0.25in}
\setlength{\evensidemargin}{0pt} \setlength{\textwidth}{6.5in}
\def\singlespace{\baselineskip=1em}
\def\doublespace{\baselineskip=2em}

\def\blank#1{$\underline{\hbox to #1{\hfil}}$}

\begin{document}
%\begin{flushright}
%\today
%\end{flushright}

\begin{center}
{\bf CMPSC 623 Problem Set 7. } \\
{\bf by Prof. Honggang Zhang} \\
\end{center}
\begin{center}
{\bf Out: November 30, 2006} \\
{\bf Due: December 7, 2006, before class.} \\

\end{center}


\begin{description}

\item[Problem.]

In this problem, we consider some generalizations of the knapsack problem.

\begin{enumerate}
\item Describe how to change the knapsack algorithm described in class to deal with the case where
the value of an item can be different from its weight.

\item Consider a knapsack problem where each item, in addition to having a weight $w_i$ and a
value $v_i$, also has a size $s_i$. The knapsack has a weight capacity $W$ and also a size capacity
$S$. Give an algorithm that runs in time $O(nWS)$ for determining the maximum value subsets of the
items that can be placed in the knapsack without violating the weight capacity or the size
capacity.

\item Consider a scenario where the items to be stolen by the thief are partitioned into types
(i.e., stereos, computers, necklaces... ). Every item belongs to exactly one type. In addition to
the weight capacity of the knapsack, the thief is further constrained to be able to steal at most
one item of each type. Give an algorithm, that runs in time $O(nW)$, that determines the optimal
subset of items that the thief is able to steal. You can assume that each item comes with an
indication of which type it belongs to, and that the items start off sorted by item type. For this
subproblem, there is no longer the size consideration introduced in part (b).


\end{enumerate}


\end{description}

\end{document}
