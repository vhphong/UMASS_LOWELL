\documentclass[11pt]{article}


\setlength {\oddsidemargin}{0.5in} \setlength
{\evensidemargin}{0.5in} \setlength {\textwidth}{5.5in}


\setlength{\parindent}{0.0in} \setlength{\parskip}{12pt}
\setlength{\topmargin}{-0.35in} \setlength{\textheight}{8.5in}
\setlength{\oddsidemargin}{-0.25in}
\setlength{\evensidemargin}{0pt} \setlength{\textwidth}{6.5in}
\def\singlespace{\baselineskip=1em}
\def\doublespace{\baselineskip=2em}

\def\blank#1{$\underline{\hbox to #1{\hfil}}$}

\begin{document}
%\begin{flushright}
%\today
%\end{flushright}

\begin{center}
{\bf CMPSC 623 Problem Set 6. Solution.} \\
{\bf by Prof. Honggang Zhang} \\
\end{center}
\begin{center}
{\bf Out: November 9, 2006} \\
{\bf Due: November 16, 2006, before class.} \\

\end{center}


\begin{description}

\item[Problem 1.]
Page 398, Exercise 16.4-1. Show it is closed under inclusion, and then show it
satisfies exchange property.

Closed under inclusion, because any subset of a set in $L_k$ is a subset of S
and its size cannot be larger than k, i.e., it is still in $L_k$.

$(S,L_k)$ also satisfies exchange property, because take any two sets $i,i'$ in
$L_k$ and $|i|<|i'|$, it must be true that $i+e\subseteq L_k$ for any $e\in
i'-i$.


\item[Problem 2.]
Read the description of activity-selection problem on page 371. Define a valid
subset $S'$ of $S$ to be a set in which all activities are mutually compatible.
For example, $S'=\{a_3, a_9, a_{11}\}$ is a valid subset. Define $I$ as the
collection of all those valid subsets of $S$. Is $(S,I)$ a subset system that
is closed under inclusion? Is it a matroid? Justify your answer.

Yes, $(S,I)$ a subset system that is closed under inclusion because any subset
of a set in $I$ will only include compatible activities.

No, it is not a matriod. For example, consider valid subsets $S_1=\{a_3, a_9,
a_{11}\}$ and $S_2=\{ a_1, a_4, a_8, a_{11} \}$. $|S_2|>|S_1|$. Take $a_1\in
S_2-S_1$, and we find that $a_1 + S_1$ is not a valid subset, thus, exchange
property cannot be satisfied.


\item[Problem 3.]
Page 398, Exercise 16.4-4. Show it is closed under inclusion, and then show it
satisfies exchange property.

First we note that $S$ is a finite nonempty set and $I \subseteq
2^S$. Consider $i\in I$ and suppose that $j\subset i$. As $j\cap
S_n\subset i \cap S_n$ for $n=1,2,...,k$, and $|j\cap S_n| \le
|i\cap S_n|\le 1$ for all $n=1,2,...,k$. So $I$ is closed under
inclusion. It follows that $(S,I)$ is a subset system.

For a valid set $i$, define $i_n=i\cap S_n$, for $n=1,2,...,k$.
For each $i_n$, $|i_n|\le 1$; $i_n \subset i$, so $i_n\in I$, but
$i_n\subset S_n$, so $|i_n \cap S_j|=0$ for $j\ne n$. Thus, for
each valid set $i$ there are indices $n_1, n_2, ..., n_r$ such
that $i=\cup^{r}_{j=1} i_{n_j}$, with $i_{n_j}\ne \emptyset$.
These $i_{n_k}$s are disjoint and $|i_{n_k}|=1; 0<|i_{n_k}|\le 1
$. It follows that $r=|i|$.

Now suppose that $i,j\in I$ and $|i|<|j|$. Then $i=\cup^q_{k=1}
i_{n_k}$ and $j=\cup^r_{k=1}j_{m_k}$, with $q<r$. So there is an
$m_l$ such that $i_{m_l}=\emptyset$ and $j_{m_l}\subset j\cap
i^c$. Thus, $i\cup j_{m_l} \in I$. The exchange property follows.
$j_{m_l}$ contains only one element.

\end{description}

\end{document}
