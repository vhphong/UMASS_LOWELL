\documentclass[11pt]{article}

\setlength {\oddsidemargin}{0.5in} \setlength
{\evensidemargin}{0.5in} \setlength {\textwidth}{5.5in}


\setlength{\parindent}{0.0in} \setlength{\parskip}{12pt}
\setlength{\topmargin}{-0.35in} \setlength{\textheight}{8.5in}
\setlength{\oddsidemargin}{-0.25in}
\setlength{\evensidemargin}{0pt} \setlength{\textwidth}{6.5in}
\def\singlespace{\baselineskip=1em}
\def\doublespace{\baselineskip=2em}

\def\blank#1{$\underline{\hbox to #1{\hfil}}$}

\begin{document}
%\begin{flushright}
%\today
%\end{flushright}

\begin{center}
{\bf CMPSC 623 Problem Set 2} \\
{\bf by Honggang Zhang} \\
\end{center}
\begin{center}
{\bf Out: September 20, 2007} \\
{\bf Due: September 27, 2007, before class.} \\

\end{center}

\begin{description}

\item[Problem 1.] Solve the following recurrences:
\begin{enumerate}
\item $T(n)=T(n-2)+n$
\item $T(n)=T(n-1) + n$
\item $T(n)=2T(n/2)+n^3$
\item $T(n)=16T(n/4)+n^3$
\item $T(n)=2T(n/16)+\sqrt{n}$
\end{enumerate}

\item[Problem 2.] A sorting algorithm is thought of as stable if all
equal elements are in the same relative order in the sorted sequence as in the original sequence.
Which of insertion sort, quick sort, and merge sort are stable, and which are unstable? Give a
simple fix to make the unstable sorts stable.

\item[Problem 3.] Use the version of the master theorem stated in
Exercise 4.4-2 (page 84) to solve the recurrence $T(n)=2T(n/2)+n\lg n$. (You need not prove
Exercise 4.4-2.)

\item[Problem 4.] Suppose we run QuickSort (not randomized version) on an array of size $n$.
Assume that each recursive partition function call divides its input array of size $k$ into two
sub-arrays, one with size $\lfloor \frac{k}{20} \rfloor$ and the other one with size
$\lceil\frac{19k}{20}\rceil-1$. What is the running time (as a function of $n$) of this QuickSort?
Why?

\item[Problem 5.]Problem 28.2-6 on page 741.

\item[Problem 6.](Divide-and-Conquer) Suppose that you can multiply two $4\times 4$ matrices using $k$
multiplications, then do the following:
\begin{enumerate}
\item Design a divide-and-conquer algorithm to multiply two $n\times n$ matrices.
\item In addition, suppose that you want to beat Strassen's algorithm by showing that your
algorithm's running time $T(n)$ is $o(n^{lg 7})$, then what is the largest $k$ you can choose?
Recall that, as we discussed in class, the larger $k$ is, the longer time your algorithm will take.
\end{enumerate}

\item[Problem 7.] (Probabilistic Analysis) Use indicator random variables to compute the expected
value of the sum of $n$ dice.

\item[Problem 8.] (Probabilistic Analysis) Page 118, Problem 5-2. Only do part (a), (b), (c).

\item[Problem 9.] Give a $\Theta(n)$-time, randomized procedure
that takes as input an array $A[1...n]$ and performs a random permutation on the array elements.
You need to write pseudo code.

\item[Problem 10.] Page 160, Problem 7-2. Only do part (a), (b), (c), and (e). Note, you need to
use substitution method for (e), and you can use the result in part (d) without proof.

%1 edition, 8.4-1
%\item[Problem 9.] exercise 7.4-2 on page 159.

%1 edition, P169 8-4
%\item[Problem 10.] Problem 7-4 on page 162.

%8.3-1 in 1st edition
%\item[Problem 6.] Exercise 7.3-1 on page 154.


\end{description}

\end{document}
