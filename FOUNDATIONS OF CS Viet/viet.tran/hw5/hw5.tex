\documentclass{scrartcl}
\usepackage{mathtools}
\usepackage{color}
\usepackage{tikz}
\usepackage{amsmath}
\usepackage{amssymb}
\usepackage{qtree}
\usepackage{forest}

\usetikzlibrary{arrows,automata}
\newcommand\Vtextvisiblespace[1][.8em]{%
  \makebox[#1]{%
    \kern.07em
    \vrule height.3ex
    \hrulefill
    \vrule height.3ex
    \kern.07em
  }% <-- don't forget this one!
}

\begin{document}
\noindent Dec 01, 2017 \\
COMP.3040 Foundation of Computer Science\\
Viet Tran\\vtran1@student.uml.edu \\
 Homework V Solution
\\

\begin{enumerate}
	\item[4.2] Consider the problem of determining whether a DFA and a regular expression are equivalent. Express this problem as a language and show that it is decidable. \\ \\
	To begin, have language $LM_{DFA,REX}$ = \{$\langle$ Q,R$\rangle$ $|$ where Q is a DFA, R is a regular expression, and L(Q) = L(R)\}. The following Turing Machine F will decide $LM_{DFA,REX}.$\\
	
	F = Input $\langle Q, R \rangle$: \\ 
	\begin{enumerate}
		\item[1.] Take the regular expression R and convert it to an equivalent DFA S using Theorem 1.28.
		
		\item[2.] Take TM C to decide $LM_{DFA}$ from Theorem 4.5 for the input $\langle Q,S \rangle$.
		\item[3.] If R accepts then accept, otherwise reject.\\
	\end{enumerate}
	
	\item[4.3] Let $ALL_DFA$ = \{$\langle A\rangle$$|$ A is a DFA and L(A) = $\Sigma$*\}. Show that $ALL_{DFA}$ is decidable.\\
	
	Consider $ALL_{DFA}$ = \{$\langle A\rangle$ where A is a DFA and recognizes $\Sigma$*\}. Turing machine M will determine $ALL_{DFA}$.\\ \\
	M = From input of $\langle A\rangle$, A being defined as a DFA:
	\begin{enumerate}
		\item[1.] Create a DFA C recognizing $\overline{L(A)}$ from Exercise 1.10.
		
		\item[2.] Execute the Turing Machine defined in Theorem 4.4 on input $\langle B\rangle$ with the condition that T will determine $E_{DFA}$.
		\item[3.] If T accepts then accept, otherwise reject.\\
	\end{enumerate}
	
	\item[4.4] Let $A\varepsilon_{CFG}$ = \{$\langle G\rangle$$|$ G is a CFG that generates $\varepsilon$\}. Show that $A\varepsilon_{CFG}$ is decidable.\\
	
	Consider $A\varepsilon_{CFG}$ = \{$\langle G\rangle$ where G is a CFG that decides $\varepsilon$\}. Turing machine M will determine $A\varepsilon_{CFG}$.\\ \\
	M = From input of $\langle G\rangle$, G being defined as a CFG:
	\begin{enumerate}
		\item[1.] Execute Turing Machine S from Theorem 4.6 on the input $\langle G, \varepsilon \rangle$, and S will decide $A_{CFG}$.
		
		\item[2.] If S accepts then accept, otherwise reject.\\

	\end{enumerate}
	\item[4.6] Let X be the set \{1, 2, 3, 4, 5\} and Y be the set \{6, 7, 8, 9, 10\}. We describe the functions $f : X\rightarrow Y$ and $g : X\rightarrow Y$ in the following tables. Answer each part and give a reason for each negative answer.\\
		\begin{table}[htb!]
			\centering
			\caption{Function f}
			\label{my-label}
			\begin{tabular}{|l|l|}
			\hline
				\textit{n} & \textit{f(n)} \\ \hline
				1 & 6 \\ \hline
				2 & 7 \\ \hline
				3 & 6 \\ \hline
				4 & 7 \\ \hline
				5 & 6 \\ \hline
			\end{tabular}
		\end{table}
		
		\begin{table}[htb!]
			\centering
			\caption{Function g}
			\label{my-label}
			\begin{tabular}{|l|l|}
			\hline
				\textit{n} & \textit{g(n)} \\ \hline
				1 & 10 \\ \hline
				2 & 9 \\ \hline
				3 & 8 \\ \hline
				4 & 7 \\ \hline
				5 & 6 \\ \hline
			\end{tabular}
		\end{table}
		\begin{enumerate}
			\item[a.] Is f one-to-one?\\
								No, because to be a one-to-one every input must have an unique output. In other words each of the input cannot have overlapping outputs. For f, f(1), f(3), and f(5) = 6. Also for f(2) and f(4) = 7. Each input have overlapping outputs or relates to the same output.
			\item[b.] Is f onto? \\ 
								No, since an onto function needs to satisfy the condition of for each and every memeber of y$\in$Y there must exist a matching member x$\in$X. But f(1), f(3), and f(5) = 6. Also f(2) and f(4) = 7. Y's 8, 9, and 10 have no X elements bounded to them. 
			\item[c.] Is f a correspondence? \\ 
								No, because a corresponding function is one that must both an one-to-one function and an onto function. The function f in this case were not either of the two, thus f cannot be a correspondence function. 
			\item[d.] Is g a one-to-one?\\
								Yes, because each element in X has their own unique output in Y. 
			\item[e.] Is g onto?\\
								Yes, because each element in Y is bounded to an element in X. There are no Y element that are not left out unlike function f which left out 8,9 and 10. 
			\item[f.] Is g a correspondence? \\
								Yes, because g is both an one-to-one and an onto function. Therefore it is also a correspondence function.
			
		\end{enumerate}
	\item[4.7] Let B be the set of all infinite sequences over \{0,1\}. Show that B is uncountable using a proof by diagonalization.\\ \\ 
						For starters, assume B is countable and there exists a correspondence $f: N\rightarrow B$. We now create x in B in a manner where it it will not be paired with anything in N. Then consider $x = x_1 , x_2 , ...$. Then let $x_i = 0$ for the cases where $f(i)_i = 1, and x_i = 1.$ if $f(i)_i = 0$ given the fact that $f(i)_i$ is the ith bit of $f(i)$. Thus, this confirming the fact that x is not f(i) for all i since it would differ from f(i) in the ith symbol. As a result, this will cause a contradictiong and proving that B is uncountable. 
	\item[4.8] Let T = \{(i, j, k)$|$ i, j, k $\in$ N\}. Show that T is countable.\\ 
	To determine if T is countable we check if it is a one-to-one function. For example, consider the case f(i,j,k) = $1^i, 3^j, 5^k$. This is a one-to-one because if $a\neq b$, then $f(a)\neq f(b)$. Thus, making T countable since it is a one-to-one. 
	
	\item[5.1] Show that $EQ_{CFG}$ is undecidable.\\
						 Consider the contradiction where $EQ_{CFG}$ is decidable. First, create a decider M for $ALL_{CFG}$ = \{$\langle G\rangle$ where G is a CFG and $L(G) = \Sigma *$. \\ 
						M = Input $\langle G\rangle$.\\ 
						\begin{enumerate}
							\item[1.] Make a CFG H with the condition $L(H) = \Sigma *$.
							\item[2.] Execute decider for $EQ_{CFG}$ for inputs $\langle G,H\rangle$.
							\item[3.] If the decider accepts then accept, otherwise reject.
							
						\end{enumerate}
						M will determine $ALL_{CFG}$ with the assumption that the decider exist for $EQ_{CFG}$. Due to $ALL_{CFG}$ being undecidable, there is a contradiction. Hence, $EQ_{CFG}$ is undecidable. 
	\item[5.2] Show that $EQ_{CFG}$ is co-Turing-recognizable.\\
						
						 To do so, consider the Turing Machine, A that recognizes the complement of $EQ_{CFG}$\\
						A = Input $\langle G, H\rangle$.\\ 
						\begin{enumerate}
							\item[1.] Generate the strings $x\in \Sigma *$ alphabetically.
							\item[2.] Consider the cases for each string x
							\item[3.] Check if $x\in L(G)$ and if $x\in L(H)$ by using $A_{CFG}$'s algorithm. 
							\item[4.] If one of the cases accepts and the other rejects then accept. Anything else just continue running. 
							
						\end{enumerate}
	\item[5.3] Find a match in the following instance of the Post Correspondence Problem \\ 
	\begin{center}
	\fontsize{18pt}{20pt}\selectfont
	\{[$\frac{ab}{abab}, \frac{b}{a}, \frac{aba}{b}, \frac{aa}{a}]$\}
	
	\end{center}
	A possible match:\\
	\begin{center}
	\fontsize{18pt}{20pt}\selectfont
	[$\frac{ab}{abab},\frac{ab}{abab}, \frac{aba}{b}, \frac{b}{a},\frac{b}{a}, \frac{aa}{a},\frac{aa}{a}]$
	
	\end{center}
	\item[5.4] If $A \leq_m B$ and B is a regular language, does that imply that A is a regular language?
Why or why not?\\\\
						No, the given condition fails to imply that A is regular. Take the example \\ \{$a^n b^n c^n | n\geq 0$\} $\leq_m \{a^n b^n | n\geq 0\}.$ The reduction of this will first test if the input provided is a member of \{$a^n b^n c^n | n\geq 0$\}. If it is then it will ouput the string $ab$. Otherwise, it will output the string $a$. 
	
	
	
\end{enumerate}

\end{document}