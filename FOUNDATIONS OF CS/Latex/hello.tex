\documentclass{article}
\usepackage{color}
\usepackage{amsmath}
\begin{document}


Hello, world! \\

To compile this latex file into a pdf file\\

\textit{
$>$pdflatex hello.tex
}
\\
\\
\textbf{this is bold text} \\

\textit{this is italicized} \\
\underline{underlined text} \\
\textbf{\textit{bold \& italicized}} \\

Use the \texttt{color} package 
to change the color of elements in \LaTeX.
 
\noindent
{\color{blue} \rule{\linewidth}{0.1mm} }

\begin{itemize}
\color{blue}
\item First item
\item Second item
\end{itemize}
 
\noindent
{\color{red} \rule{\linewidth}{0.2mm} }

Here is an example of a language definition \\
$
\sum = \{0, 1 \}  \\
$
\{ w is any string of the form $00^*1111^*0$ \} \\

A DFA M may be defined as \\

% begin math mode with leading \$

M = \{Q, $\sum$, $\delta$, $q_0$, F \} \\
where Q is the set of states \\
$\sum$ is the set of symbols in the alphabet \\
$\delta$ is the transition function \\
$q_0$ is the start state \\
F $\subseteq$ Q is the set of final states \\

% end math mode with trailing \$

For example \\

% begin math mode with leading \$

Q = \{$q_0$, $q_1$, $q_2$\} \\
$\sum$ = \{0, 1\} \\
F = \{$q_2$\} \\
$q_0$ = $q_0$ \\
$\delta = \{((q_0, 0), q_1), ((q_0, 1), q_0), ((q_1, 0), q_2), ((q_1, 1), q_1)\}$ \\

% end math mode with trailing \$

In Table form: \\

\begin{center}
 \begin{tabular}{||c c c||} 
 \hline
 \space 	& 0 		& 1 		\\ [0.5ex] 
 \hline
 $q_0$ 		& $q_0$ 	& $q_1$		\\
 \hline
 $q_1$ 		& $q_2$ 	& $q_1$		\\
 \hline
 $q_2$ 		& $q_0$ 	& $q_2$		\\
 \hline
\end{tabular}
\end{center}



\end{document} 

